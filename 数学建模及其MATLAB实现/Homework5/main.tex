\documentclass{article}
\usepackage{fancyhdr}
\usepackage{ctex}
\usepackage{listings}
\usepackage{graphicx}
\usepackage[a4paper, body={18cm,22cm}]{geometry}
\usepackage{amsmath,amssymb,amstext,enumerate,graphicx}
\usepackage{float,abstract,booktabs,indentfirst,amsmath}
\usepackage{array}
\usepackage{booktabs}
\usepackage{multirow}
\usepackage{url}
\usepackage{diagbox}
\renewcommand\arraystretch{1.4}
\usepackage{indentfirst}
\setlength{\parindent}{2em}
\usepackage{enumerate}
\setmonofont{Consolas}
\usepackage{listings}
\usepackage{xcolor}
\usepackage{makecell}
\usepackage{enumitem}
\usepackage{tikz}
\usepackage{wrapfig}
\usepackage{tkz-euclide}
\usepackage{pgfplots}
\setfontfamily{\timesfont}{Times New Roman}

\begin{document}

\newcommand{\titem}[1]{
~\\
\begin{itemize}
    \item \heiti \large {#1}
\end{itemize}
}

\newcommand{\bb}[1]{{\heiti {#1}}}

\renewcommand{\d}{\mathrm{d}}

\newcommand{\cf}[1]{$^{#1}\textrm{C}$}

\title{《数学建模及其MATLAB实现》第五次课程作业}
\author{李鹏达}
    

\begin{center}
    \LARGE \textbf{\heiti 《数学建模及其{\timesfont MATLAB}实现》第五次课程作业} \\[0.5em]
    \large 李鹏达 10225101460
\end{center}

考古专家如何判定一件文物的年代?收藏家如何鉴别一件画作、字帖、艺术品等的真伪?
通常使用放射性元素测年方法.

请做一个调研,寻找一个艺术品或文物鉴别的案例,然后详细分析其中的微分方程建模及求解过程.

\titem{案例}

2021年3月23日上午,据央视新闻消息,四川省文物考古研究院联合北京大学对6个坑的73份炭屑样品使用碳14年代检测方法进行了分析,对年代分布区间进行了初步判定。其中K4坑年代最有可能是在公元前1199年至公元前1017年,也就是距今约3200年至3000年,这就印证了三星堆新发现的4号坑碳14检测的年代区间属于商代晚期。

\titem{分析}

\cf{14} 测年法的原理基于 \cf{14} 同位素的放射性衰变. \cf{14}测年法利用\cf{14}的放射性衰变原理来测定有机物的年龄. \cf{14}在生物体内与\cf{12}保持一定比例,但在生物死亡后,\cf{14}开始衰变成 $^{14}\textrm{N}$.通过测量样品中\cf{14}与\cf{12}的比值,结合\cf{14}的半衰期,可以推算出样品的年龄.

放射性同位素衰变时,单位时间内衰变的同位素数量与同位素的数量成正比,这个关系可以用微分方程表示为
\begin{equation*}
    \frac{\d N}{\d t} = -\lambda N
\end{equation*}

其中, $N$ 为同位素的数量,$\lambda$ 为衰变常数.

设 $N_0$ 为初始时刻的同位素数量,$N(t)$ 为 $t$ 时刻的同位素数量,那么该微分方程的解为
\begin{equation*}
    N(t) = N_0 e^{-\lambda t}
\end{equation*}

对于 \cf{14},其半衰期为 5730 年,即 
$$
\begin{aligned}
    N(5730) &= N_0 e^{-5730\lambda} = \frac{1}{2} N_0 \\
    \lambda &= \frac{\ln 2}{5730}
\end{aligned}
$$

假设被测样品的 \cf{14}/\cf{12} 原始比值与大气中的 \cf{14}/\cf{12} 比值 $r$ 相同,即 $N_{0}/N_{12} = r$,那么
$$
\begin{aligned}
    N(t) &= N_0 e^{-\lambda t} \\
    &= r N_{12} e^{-\lambda t} \\
\end{aligned}
$$

即
$$
\begin{aligned}
    \frac{N(t)}{N_{12}} &= r e^{-\lambda  t} \\
    &= r e^{-\lambda t} \\
    t &= -\frac{1}{\lambda} \ln \frac{N(t)}{r N_{12}} \\
    &= \frac{5730}{\ln 2} \ln \frac{r N_{12}}{N(t)}
\end{aligned}
$$

因此,通过测量样品中 \cf{14}/\cf{12} 的比值,结合大气中的 \cf{14}/\cf{12} 比值,可以推算出样品的年龄.

\end{document}