\documentclass{article}
\usepackage{fancyhdr}
\usepackage{ctex}
\usepackage{listings}
\usepackage[a4paper, body={18cm,22cm}]{geometry}
\usepackage{amsmath,amssymb,amstext,wasysym,enumerate,graphicx}
\usepackage{float,abstract,booktabs,indentfirst,amsmath}
\usepackage{multirow}
\usepackage{enumitem}
\usepackage{listings}
\usepackage{xcolor}
\usepackage{tabularx}
\usepackage{subfigure}
\usepackage[xetex,pagebackref]{hyperref}
\usepackage[most]{tcolorbox}
\usepackage{accsupp}
\newcommand\emptyaccsupp[1]{\BeginAccSupp{ActualText={}}#1\EndAccSupp{}}
\setlength{\parindent}{2em}
\renewcommand\arraystretch{1.4}
\setmonofont{DejaVu Sans Mono}
\setCJKmonofont{黑体}
% \usepackage{array}
% \usepackage{booktabs}
% \usepackage{url}
% \usepackage{diagbox}
% \usepackage{makecell}
% \usepackage{tikz}
% \usepackage{subcaption}
% \usetikzlibrary{positioning, arrows.meta}
\hypersetup{CJKbookmarks=true,colorlinks=true,citecolor=blue,%
            linkcolor=blue,urlcolor=blue,bookmarksnumbered=true,%
            bookmarksopen=true,breaklinks=true}
\lstset{
    % language = C,
    xleftmargin = 3em,xrightmargin = 3em, aboveskip = 1em,
	backgroundcolor = \color{white}, % 背景色
	basicstyle = \small\ttfamily, % 基本样式 + 小号字体
	rulesepcolor= \color{gray}, % 代码块边框颜色
	breaklines = true, % 代码过长则换行
	numbers = left, % 行号在左侧显示
	numberstyle = \small\emptyaccsupp, % 行号字体
    numbersep = -14pt, 
    keywordstyle=\color{purple}\bfseries, % 关键字颜色
    commentstyle =\color{red!50!green!50!blue!60}, % 注释颜色
    stringstyle = \color{red}, % 字符串颜色
    morekeywords={ASSERT, int64_t, uint32_t},
	frame = shadowbox, % 用(带影子效果)方框框住代码块
	showspaces = false, % 不显示空格
    showstringspaces = false,
	columns = fixed, % 字间距固定
    literate=
        {^+}{{{\color{black}\textbf{+}}\colorbox{green!30}{\phantom{XX}}}}1
        {+\t}{{{\color{black}\textbf{+}}\colorbox{green!30}{\phantom{XX}}}}1,
}
\newcommand{\notebox}[2][NOTE]{%
    \begin{tcolorbox}[
        colback=cyan!5!white,          
        colframe=cyan!50!black,       
        fonttitle=\bfseries\large,   
        coltitle=white,               
        title=#1,
        rounded corners,              
        boxrule=1.5pt,                
        width=\textwidth,            
        drop shadow=black!30,         
        enhanced,                  
    ]
    #2
    \end{tcolorbox}
}
\newenvironment{enum}{
    \begin{enumerate}[label=(\arabic*), noitemsep]
}{
    \end{enumerate}
}
\newcommand{\img}[3][0.9]{%
    \begin{figure}[H]
        \centering
        \includegraphics[width=#1\textwidth]{img/#2.png}
        \caption{#3}
    \end{figure}
}
\newcommand{\imgtwo}[4][0.45]{%
    \begin{figure}[H]
        \centering
        \subfigure{
            \includegraphics[width=#1\textwidth]{img/#2.png}
        }
        \subfigure{
            \includegraphics[width=#1\textwidth]{img/#3.png}
        }
        \caption{#4}
    \end{figure}
}
\newcommand{\imgthree}[6][0.9]{%
    \begin{figure}[H]
        \centering
        \includegraphics[width=#1\textwidth]{img/#3.png}
        \subfigure{
            \includegraphics[width=#2\textwidth]{img/#4.png}
            \includegraphics[width=#2\textwidth]{img/#5.png}
        }
        \caption{#6}
    \end{figure}
}
\newcommand{\subsubsubsection}[1]{\paragraph{#1}\mbox{}}
\setcounter{secnumdepth}{4} 
\setcounter{tocdepth}{4}
%--------------------页眉--------------------%
\pagestyle{fancy}
\fancyhead[L]{}
\fancyhead[R]{}
\fancyhead[C]{华东师范大学软件工程学院实验报告}
\fancyfoot[C]{-\thepage-}
\renewcommand{\headrulewidth}{1.5pt}
%--------------------标题--------------------%
\begin{document}

\newcommand{\courseName}{Cloud Computing}
\newcommand{\labName}{Headoop 搭建}
\newcommand{\studentName}{李鹏达}
\newcommand{\studentID}{10225101460}
\newcommand{\grade}{2022级}
\newcommand{\labNo}{No. 3}
\newcommand{\labDate}{2024年12月18日}
\newcommand{\labTime}{第15 - 16周}

\begin{center}
    \LARGE{{\textbf{\heiti 华东师范大学软件工程学院实验报告}}}
    \begin{table}[H]
        \centering
        \begin{tabular}{cp{3cm}<{\centering}ccp{3cm}<{\centering}ccp{3.5cm}<{\centering}}
            实验课程:    & \courseName & \quad & 年\qquad 级: & \grade & \quad & 实验成绩: &  \\
            \cline{2-2} \cline{5-5} \cline{8-8}
            实验名称:    & \labName & \quad & 姓\qquad 名:    & \studentName & \quad & 实验日期: & \labDate
            \\ \cline{2-2} \cline{5-5} \cline{8-8}
            实验编号: &   \labNo   & \quad & 学\qquad 号: & \studentID & \quad & 实验时间: & \labTime\\ \cline{2-2} \cline{5-5} \cline{8-8}
        \end{tabular}
    \end{table}
\end{center}
\rule{\textwidth}{1pt}
%--------------------正文--------------------%
\section{实验内容}

\begin{itemize}[noitemsep]
    \item Linux系统安装及配置
    \item Hadoop单例模式搭建
    \item Hadoop伪分布式模式搭建
    \item 虚拟机克隆及相关网络配置
    \item 集群时间同步
    \item Hadoop集群模式部署
    \item MapReduce案例应用
\end{itemize}

\section{实验关键步骤}

\subsection{Linux 系统的安装及配置}

\subsubsection{安装Linux系统}

\subsubsubsection{创建虚拟机}

\notebox{
    由于在实验一中已经安装了Oracle VirtualBox虚拟机平台,因此,我选择使用VirtualBox创建虚拟机,而不是实验手册中的VMware Workstation。
}

\begin{enum}
    \item 新建一个虚拟机,名称为 \texttt{lipengda001}
    \item 将虚拟光盘选择为预先下载好的 \texttt{CentOS7} 映像,勾选“跳过自动安装”。
    \item 为虚拟机分配 2GB 内存,20GB 硬盘空间。
\end{enum}

\imgthree{0.45}{1.1.1.1}{1.1.1.2}{1.1.1.3}{创建虚拟机}

\subsubsubsection{启动虚拟机}

启动虚拟机,选择 \texttt{Install CentOS Linux 7} 安装系统。

\img[0.5]{1.1.2.1}{安装系统(1)}

在软件选择中,选择 “最小安装”。

\notebox{
    为减少资源占用并加快安装速度,我在此处选择了“最小安装”,而不是“带GUI的服务器”。
}

\img[0.5]{1.1.2.2}{安装系统(2)}

设置 root 账户密码。

\img[0.5]{1.1.2.3}{安装系统(3)}

等待系统安装完成。

\subsubsection{Linux 系统相关配置}

\subsubsubsection{网络配置}

\begin{enum}
    \item 输入命令 \texttt{ip addr} 查看网络配置。
    
    \begin{lstlisting}[language=bash]
    ip addr
    \end{lstlisting}

    \img{1.2.1.1}{查看网络配置}

    可以看到,我的网卡名称叫 \texttt{enp0s3}。

    \item 查看网卡 IP 信息的配置文件。
    
    \begin{lstlisting}[language=bash]
    ls /etc/sysconfig/network-scripts/
    \end{lstlisting}

    \img{1.2.1.2}{查看配置文件}

    可以看到,网卡配置文件为 \texttt{ifcfg-enp0s3}。

    \item 查看网卡配置文件。
    
    \begin{lstlisting}[language=bash]
    cat /etc/sysconfig/network-scripts/ifcfg-enp0s3
    \end{lstlisting}

    \img{1.2.1.3}{查看网卡配置文件}

    \item 修改网卡配置文件,将 \texttt{ONBOOT} 的值改为 \texttt{yes}。
    
    \begin{lstlisting}[language=bash]
    vi /etc/sysconfig/network-scripts/ifcfg-enp0s3
    \end{lstlisting}

    \img{1.2.1.4}{修改网卡配置文件}

    \item 重启网络服务。
    
    \begin{lstlisting}[language=bash]
    service network restart
    \end{lstlisting}

    \item 再次查看网络配置。
    
    \begin{lstlisting}[language=bash]
    ip addr
    \end{lstlisting}

    \img{1.2.1.5}{查看网络配置}

    可以看到,分配的 IP 地址为 \texttt{192.168.1.111}。

    \item 再次修改网卡配置文件,将 \texttt{BOOTPROTO} 的值改为 \texttt{static},并添加 \texttt{IPADDR}、\texttt{NETMASK}、\texttt{GATEWAY} 等字段,将IP地址设置为 \texttt{192.168.1.111}。
    
    \begin{lstlisting}[language=bash]
    vi /etc/sysconfig/network-scripts/ifcfg-enp0s3
    \end{lstlisting}

    \img[0.8]{1.2.1.6}{修改网卡配置文件}

    \item 重启网络服务。
    
    \begin{lstlisting}[language=bash]
    service network restart
    \end{lstlisting}

    \item 查看网络配置,确认配置成功。
    
    \begin{lstlisting}[language=bash]
    ip addr
    \end{lstlisting}

    \img[0.8]{1.2.1.7}{查看网络配置}

    \item 测试网络连通性。
    
    \begin{lstlisting}[language=bash]
    ping 192.168.1.111
    \end{lstlisting}

    \img{1.2.1.8}{测试网络连通性}
\end{enum}

\subsubsubsection{修改主机名}

\begin{enum}
    \item 查看主机名。
    
    \begin{lstlisting}[language=bash]
    hostname
    \end{lstlisting}

    \img{1.2.2.1}{查看主机名}

    \item 修改主机名。
    
    \begin{lstlisting}[language=bash]
    vi /etc/sysconfig/network
    \end{lstlisting}

    设置 \texttt{NETWORKING} 为 \texttt{yes},\texttt{HOSTNAME} 为 \texttt{lipengda001}。

    \img{1.2.2.2}{修改主机名}

    \begin{lstlisting}[language=bash]
    hostnamectl set-hostname lipengda001
    \end{lstlisting}

    \notebox{
        此处实验手册有误,在 \texttt{CentOS7} 中,\texttt{/etc/sysconfig/network} 文件中的 \texttt{HOSTNAME} 字段已经被废弃。该项配置已经被移动到 \texttt{/etc/hostname} 文件中。使用 \texttt{hostnamectl} 命令修改主机名是一个更好的选择。

        \img{1.2.2.3}{\texttt{/usr/share/doc/initscripts-9.49.53/sysconfig.txt}中的说明}
    }

    \item 重启虚拟机。
    
    \begin{lstlisting}[language=bash]
    reboot
    \end{lstlisting}

    \item 查看主机名是否修改成功。

    \begin{lstlisting}[language=bash]
    hostname
    \end{lstlisting}

    使用 \texttt{ping} 命令进行验证。

    \begin{lstlisting}[language=bash]
    ping lipengda001
    \end{lstlisting}

    \img{1.2.2.4}{验证主机名修改成功}
\end{enum}

\subsubsubsection{建立IP地址与虚拟机名称的对应关系}

修改域名解析映射文件,使得后续可以直接通过主机名访问。

\begin{lstlisting}[language=bash]
    vi /etc/hosts
\end{lstlisting}

在文件中添加 \texttt{192.168.1.111 lipengda001}。

\img{1.2.3.1}{修改域名解析映射文件}

\subsubsubsection{Linux系统与Windows系统进行网络通讯}

\begin{enum}
    \item 在 Windos 系统中,使用 \texttt{ping} 命令测试与 Linux 系统的网络连通性。
    
    \begin{lstlisting}[language=bash]
    ping 192.168.1.111
    \end{lstlisting}

    \img{1.2.4.1}{Windows系统与Linux系统网络连通性测试}

    \item 在 Windos 端,尝试 \texttt{ping} 主机名 \texttt{lipengda001}。
    
    \begin{lstlisting}[language=bash]
    ping lipengda001
    \end{lstlisting}

    显示无法解析主机名。

    \img{1.2.4.2}{Windows系统无法解析主机名}

    因此,需要在 Windows 系统中修改 \texttt{hosts} 文件(位于
    \texttt{C:\textbackslash Windows\textbackslash System32\textbackslash drivers\textbackslash etc}),添加 \texttt{192.168.1.111 lipengda001}。

    \img{1.2.4.3}{修改Windows系统hosts文件}

    再次尝试 \texttt{ping} 主机名 \texttt{lipengda001}。

    \begin{lstlisting}[language=bash]
    ping lipengda001
    \end{lstlisting}

    \img{0.1.2.4.4}{Windows系统解析主机名成功}
\end{enum}

\subsection{Hadoop 单例模式搭建}

\subsubsection{安装JDK}

\subsubsubsection{下载解压}

\begin{enum}
    \item 创建一个文件夹 \texttt{app}
    
    \begin{lstlisting}[language=bash]
    mkdir app
    \end{lstlisting}

    \item 检查系统是否存在openjdk
    
    \begin{lstlisting}[language=bash]
    rpm -qa | grep java
    \end{lstlisting}

    \img{2.1.1.1}{检查系统是否存在openjdk}

    此处没有输出,说明系统中没有安装openjdk。

    \item 下载JDK
    
    \begin{lstlisting}[language=bash]
    curl -L -C - -b "oraclelicense=accept-securebackup-cookie" -O http://download.oracle.com/otn-pub/java/jdk/8u131-b11/d54c1d3a095b4ff2b6607d096fa80163/jdk-8u131-linux-x64.tar.gz
    \end{lstlisting}

    \img{2.1.1.2}{下载JDK}

    \item 解压JDK
    
    \begin{lstlisting}[language=bash]
    tar -zxvf jdk-8u131-linux-x64.tar.gz -C .
    \end{lstlisting}

    \img{2.1.1.3}{解压JDK}

    \item 检验JDK是否安装成功
    
    \begin{lstlisting}[language=bash]
    cd jdk1.8.0_131
    ./bin/java -version
    \end{lstlisting}

    \img{2.1.1.4}{检验JDK是否安装成功}
\end{enum}

\subsubsection{配置环境变量}

\begin{enum}
    \item 编辑 \texttt{/etc/profile} 文件
    
    \begin{lstlisting}[language=bash]
    vi /etc/profile
    \end{lstlisting}

    在文件末尾添加以下内容 

    \begin{lstlisting}[language=bash]
    export JAVA_HOME=/root/app/jdk1.8.0_131
    export PATH=$JAVA_HOME/bin:$PATH
    \end{lstlisting}

    \img[0.75]{2.2.1}{编辑/etc/profile文件}

    \item 使配置生效
    
    \begin{lstlisting}[language=bash]
    source /etc/profile
    \end{lstlisting}

    \item 检验环境变量是否配置成功
    
    \begin{lstlisting}[language=bash]
    java -version
    \end{lstlisting}

    \img{2.2.2}{检验环境变量是否配置成功}

    编写一个简单的Java程序,检验JDK是否配置成功。

    \begin{lstlisting}[language=bash]
    mkdir workspace
    cd workspace
    vi HelloWorld.java
    \end{lstlisting}

    输入以下内容

    \begin{lstlisting}[language=java]
    public class HelloWorld {
        public static void main(String[] args) {
            System.out.println("Hello World!");
        }
    }
    \end{lstlisting}

    编译运行

    \begin{lstlisting}[language=bash]
    javac HelloWorld.java
    java HelloWorld
    \end{lstlisting}

    \img{2.2.3}{Java程序运行结果}
\end{enum}

\subsubsection{安装Hadoop}

\subsubsubsection{下载Hadoop}

下载 Hadoop。

\begin{lstlisting}[language=bash]
    curl -O https://archive.apache.org/dist/hadoop/common/hadoop-2.5.0/hadoop-2.5.0.tar.gz
\end{lstlisting}

解压 Hadoop。

\begin{lstlisting}[language=bash]
    tar -zxvf hadoop-2.5.0.tar.gz -C .
\end{lstlisting}

\img{2.3.1.1}{下载解压Hadoop}

\subsubsubsection{配置环境变量}

\label{subsubsubsection:config-hadoop-env}

\begin{enum}
    \item 编辑 \texttt{/etc/profile} 文件
    
    \begin{lstlisting}[language=bash]
    vi /etc/profile
    \end{lstlisting}

    在文件末尾添加以下内容 

    \begin{lstlisting}[language=bash]
    export HADOOP_HOME=/root/app/hadoop-2.5.0
    export PATH=$HADOOP_HOME/bin:$HADOOP_HOME/sbin:$PATH
    \end{lstlisting}

    \img[0.65]{2.3.2.1}{编辑/etc/profile文件}

    \item 使配置生效
    
    \begin{lstlisting}[language=bash]
    source /etc/profile
    \end{lstlisting}

    \item 检验环境变量是否配置成功
    
    \begin{lstlisting}[language=bash]
    hadoop
    \end{lstlisting}

    \img[0.8]{2.3.2.2}{检验环境变量是否配置成功}
\end{enum}

\subsubsubsection{配置\texttt{hadoop-env.sh}}

\label{subsubsubsection:hadoop-env.sh}

修改 \texttt{hadoop-env.sh} 文件。

\begin{lstlisting}[language=bash]
    vi $HADOOP_HOME/etc/hadoop/hadoop-env.sh
\end{lstlisting}

将 \texttt{JAVA\_HOME} 配置为 \texttt{/root/app/jdk1.8.0\_131}。

\img{2.3.3.1}{修改\texttt{hadoop-env.sh}文件}

\subsubsubsection{测试}

通过单词统计案例测试 Hadoop 是否配置成功。

进入 workspace 目录。

\begin{lstlisting}[language=bash]
    cd ~/app/workspace
\end{lstlisting}

运行 Hadoop。

\begin{lstlisting}[language=bash]
    hadoop jar $HADOOP_HOME/share/hadoop/mapreduce/hadoop-mapreduce-examples-2.5.0.jar wordcount ./HelloWorld.java ./out
\end{lstlisting}

\img{2.3.4.1}{运行Hadoop测试案例}

查看结果。

\begin{lstlisting}[language=bash]
    cd out
    ls
    cat part-r-00000
\end{lstlisting}

\img{0.2.3.4.2}{查看结果}

\subsection{Hadoop 伪分布式模式搭建}

\subsubsection{修改配置文件}

进入 \texttt{Hadoop} 配置文件目录。

\begin{lstlisting}[language=bash]
    cd $HADOOP_HOME/etc/hadoop
\end{lstlisting}

\subsubsubsection{配置\texttt{hadoop-env.sh}}

在 \ref{subsubsubsection:hadoop-env.sh} 中已经配置过 \texttt{hadoop-env.sh} 文件。

\subsubsubsection{配置\texttt{core-site.xml}}

\begin{lstlisting}[language=bash]
    vi core-site.xml
\end{lstlisting}

添加以下内容。

\begin{lstlisting}[language=xml]
    <configuration>
        <property>
            <name>fs.default.name</name>
            <value>hdfs://lipengda001:9000</value>
        </property>
    </configuration>
\end{lstlisting}

\img{3.1.2.1}{配置\texttt{core-site.xml}}

\subsubsubsection{配置\texttt{hdfs-site.xml}}

\begin{enum}
    \item 配置 hadoop 公共目录。
    
    在 \texttt{Hadoop} 的配置文件目录下创建 \texttt{data} 目录。
    在 \texttt{data} 目录下创建 \texttt{namenode}、\texttt{datanode} 和 \texttt{tmp} 目录。

    \begin{lstlisting}[language=bash]
    cd $HADOOP_HOME
    mkdir data
    mkdir data/namenode
    mkdir data/datanode
    mkdir data/tmp
    \end{lstlisting}

    \img{3.1.3.1}{配置hadoop公共目录}

    \item 修改配置文件
    
    \begin{lstlisting}[language=bash]
    cd $HADOOP_HOME/etc/hadoop
    vi hdfs-site.xml
    \end{lstlisting}
    
    在 \texttt{hdfs-site.xml} 文件中添加以下内容。

    \begin{lstlisting}[language=xml]
    <configuration>
        <property>
            <name>dfs.name.dir</name>
            <value>/root/app/hadoop-2.5.0/data/namenode</value>
        </property>
        <property>
            <name>dfs.data.dir</name>
            <value>/root/app/hadoop-2.5.0/data/datanode</value>
        </property>
        <property>
            <name>dfs.tmp.dir</name>
            <value>/root/app/hadoop-2.5.0/data/tmp</value>
        </property>
        <property>
            <name>dfs.replication</name>
            <value>1</value>
        </property>
    </configuration>
    \end{lstlisting}

    \img{3.1.3.2}{修改配置文件}

\end{enum}

\subsubsubsection{配置\texttt{mapred-site.xml}}

\begin{lstlisting}[language=bash]
    mv mapred-site.xml.template mapred-site.xml
    vi mapred-site.xml
\end{lstlisting}

在 \texttt{mapred-site.xml} 文件中添加以下内容。

\begin{lstlisting}[language=xml]
    <configuration>
        <property>
            <name>mapreduce.framework.name</name>
            <value>yarn</value>
        </property>
    </configuration>
\end{lstlisting}

\img{3.1.4.1}{配置\texttt{mapred-site.xml}}

\subsubsubsection{配置\texttt{yarn-site.xml}}

\begin{lstlisting}[language=bash]
    vi yarn-site.xml
\end{lstlisting}

在 \texttt{yarn-site.xml} 文件中添加以下内容。

\begin{lstlisting}[language=xml]
    <configuration>
        <property>
            <name>yarn.resourcemanager.hostname</name>
            <value>lipengda001</value>
        </property>
        <property>
            <name>yarn.nodemanager.aux-services</name>
            <value>mapreduce_shuffle</value>
        </property>
    </configuration>
\end{lstlisting}

\img{3.1.5.1}{配置\texttt{yarn-site.xml}}

\subsubsubsection{配置\texttt{slaves}}

\begin{lstlisting}[language=bash]
    vi slaves
\end{lstlisting}

将 \texttt{localhost} 改为 \texttt{lipengda001}。

\img{3.1.6.1}{配置\texttt{slaves}}

\subsubsubsection{配置环境变量 \texttt{PATH}}

在 \ref{subsubsubsection:config-hadoop-env} 中已经配置过环境变量。

\subsubsection{启动}

\subsubsubsection{namenode 格式化}

\begin{lstlisting}[language=bash]
    hdfs namenode -format
\end{lstlisting}

\img{3.2.1.1}{namenode格式化}

\subsubsubsection{启动}

\begin{lstlisting}[language=bash]
    start-dfs.sh
    start-yarn.sh
\end{lstlisting}

输入 \texttt{jps} 查看进程。

\begin{lstlisting}[language=bash]
    jps
\end{lstlisting}

出现以下进程,说明启动成功。

\begin{itemize}[noitemsep]
    \item NameNode
    \item DataNode
    \item SecondaryNameNode
    \item ResourceManager
    \item NodeManager
\end{itemize}

\img{0.3.2.2.1}{启动Hadoop}

\subsection{虚拟机克隆及相关网络配置}

\notebox{
    由于虚拟机中未安装后续步骤需要的 \texttt{ntp},为减少重复工作,先在 \texttt{lipengda001} 虚拟机中安装 \texttt{ntp}后再进行克隆。
}
    
\notebox{
    由于 CentOS7 在 2024年6月30日已经停止维护,因此 \texttt{yum} 源已经失效。为了解决这个问题,需要先将 \texttt{yum} 源更换为 \texttt{CentOS Vault} 源。
}

\begin{lstlisting}[language=bash]
    sed -i s/^#.*baseurl=http/baseurl=http/g /etc/yum.repos.d/*.repo
    sed -i s/^mirrorlist=http/#mirrorlist=http/g /etc/yum.repos.d/*.repo
    sed -i s/mirror.centos.org/vault.centos.org/g /etc/yum.repos.d/*.repo
    yum clean all
    yum makecache
\end{lstlisting}

\begin{lstlisting}[language=bash]
    yum install -y ntp
\end{lstlisting}

\notebox{
    为方便后续主机名和IP地址的配置,我选择先创建一个 \texttt{sh} 脚本,将主机名和IP地址的配置写入脚本中,将来再在克隆的虚拟机中执行该脚本。
}

\begin{lstlisting}[language=bash]
    vi ~/config.sh
\end{lstlisting}

内容如下。

\begin{lstlisting}[language=bash]
    #!/bin/bash

    if [ $# -ne 2 ]; then
        echo "Usage: $0 hostname ip"
        exit 1
    fi

    hostnamectl set-hostname $1
    sed -i "s/HOSTNAME=.*/HOSTNAME=$1/g" /etc/sysconfig/network

    echo "$2 $1" >> /etc/hosts

    sed -i "s/IPADDR=.*/IPADDR=$2/g" /etc/sysconfig/network-scripts/ifcfg-enp0s3

    service network restart

    hostname
    ip addr
\end{lstlisting}

\begin{lstlisting}[language=bash]
    chmod +x ~/config.sh
\end{lstlisting}

\subsubsection{虚拟机克隆}

在 VirtualBox 中选择 \texttt{lipengda001} 虚拟机,右键,然后点击 \texttt{复制}。

\img{4.1.1}{虚拟机克隆(1)}

克隆两台虚拟机,分别命名为 \texttt{lipengda002} 和 \texttt{lipengda003}。

\imgtwo{4.1.2}{4.1.3}{虚拟机克隆(2)}

\subsubsection{相关网络配置}

在 \texttt{lipengda002} 和 \texttt{lipengda003} 虚拟机中执行 \texttt{config.sh} 脚本。

\begin{lstlisting}[language=bash]
    ./config.sh lipengda002 192.168.1.222
\end{lstlisting}

\begin{lstlisting}[language=bash]
    ./config.sh lipengda003 192.168.1.233
\end{lstlisting}

\img{4.2.1}{002相关网络配置}

\img{4.2.2}{003相关网络配置}

修改 Windows 系统中的 \texttt{hosts} 文件。

\img{4.2.3}{修改Windows系统hosts文件}

在 Windows 系统中测试网络连通性。

\begin{lstlisting}[language=bash]
    ping lipengda001
    ping lipengda002
    ping lipengda003
\end{lstlisting}

\img{4.2.4}{Windows系统测试网络连通性}


\subsection{SSH 免密码登录}

\subsubsection{创建公钥/私钥}

在 \texttt{lipengda001} 虚拟机中创建公钥/私钥。

\begin{lstlisting}[language=bash]
    cd ~/.ssh
    ssh-keygen -t rsa
\end{lstlisting}

\img[0.8]{5.1.1}{创建公钥/私钥}

在 \texttt{lipengda002} 和 \texttt{lipengda003} 虚拟机中进行同样的操作。

\subsubsection{对虚拟机自己实行免密码登录}

\begin{lstlisting}[language=bash]
    ssh-copy-id lipengda001
\end{lstlisting}

\img[0.8]{5.2.1}{对虚拟机自己实行免密码登录}

尝试免密码登录。

\begin{lstlisting}[language=bash]
    ssh lipengda001
\end{lstlisting}

\img{5.2.2}{尝试免密码登录}

在 \texttt{lipengda002} 和 \texttt{lipengda003} 虚拟机中进行同样的操作。

\subsubsection{虚拟机之间相互通讯}

\subsubsubsection{虚拟机1访问虚拟机2和3}

\begin{enum}
    \item 通过IP地址
    
    \begin{lstlisting}[language=bash]
    ssh-copy-id 192.168.1.222
    ssh 192.168.1.222
    \end{lstlisting}

    \img[0.8]{5.3.1.1}{通过IP地址访问虚拟机2}

    \begin{lstlisting}[language=bash]
    ssh-copy-id 192.168.1.233
    ssh 192.168.1.233
    \end{lstlisting}

    \img[0.8]{5.3.1.2}{通过IP地址访问虚拟机3}

    对虚拟机2和3进行同样的操作。

    \item 通过主机名
    
    编辑 \texttt{hosts} 文件。

    \begin{lstlisting}[language=bash]
    vi /etc/hosts
    \end{lstlisting}

    添加以下内容。

    \begin{lstlisting}[language=bash]
    192.168.1.222 lipengda002
    192.168.1.233 lipengda003
    \end{lstlisting}

    \img[0.8]{5.3.1.3}{编辑\texttt{hosts}文件}

    尝试使用主机名访问。

    \begin{lstlisting}[language=bash]
    ssh lipengda002
    ssh lipengda003
    \end{lstlisting}

    \img[0.8]{5.3.1.4}{通过主机名访问虚拟机2和3}
\end{enum}

在 \texttt{lipengda002} 和 \texttt{lipengda003} 虚拟机中进行同样的操作。

\subsection{集群时间同步}

\subsubsection{设置时间}

\subsubsubsection{查看虚拟机中是否安装了ntp}

\begin{lstlisting}[language=bash]
    rpm -qa | grep ntp
\end{lstlisting}

\img{6.1.1.1}{查看是否安装了ntp}

出现内容说明已经安装了ntp。

\subsubsubsection{配置ntpd}

\begin{lstlisting}[language=bash]
    vi /etc/sysconfig/ntpd
\end{lstlisting}

添加以下内容

\begin{lstlisting}[language=bash]
    SYNC_HWCLOCK=yes
\end{lstlisting}

\img{6.1.2.1}{配置ntpd}

\subsubsubsection{开启ntpd服务}

\begin{lstlisting}[language=bash]
    service ntpd status
    service ntpd start
    chkconfig ntpd on
\end{lstlisting}

\img{6.1.3.1}{开启ntpd服务}

\subsubsubsection{修改ntp配置}

\begin{lstlisting}[language=bash]
    vi /etc/ntp.conf
\end{lstlisting}

取消注释以下内容

\begin{lstlisting}[language=bash]
    #restrict 192.168.1.0 mask 255.255.255.0 nomodify notrap
\end{lstlisting}

注释以下内容

\begin{lstlisting}[language=bash]
    server 0.centos.pool.ntp.org iburst
    server 1.centos.pool.ntp.org iburst
    server 2.centos.pool.ntp.org iburst
    server 3.centos.pool.ntp.org iburst
\end{lstlisting}

在 \texttt{\#crypto} 下添加以下内容

\begin{lstlisting}[language=bash]
    server 127.127.1.0
    fudge  127.127.1.0 stratum 10
\end{lstlisting}

\notebox{
    此处实验手册遗漏了重启ntpd服务。
}

\begin{lstlisting}[language=bash]
    service ntpd restart
\end{lstlisting}

\img{6.1.4.1}{修改ntp配置}

\subsubsubsection{设置具体时间}

随便设置一个时间。

\begin{lstlisting}[language=bash]
    date -s 2024-12-01
    date -s 12:00:00
    date
\end{lstlisting}

\img{6.1.5.1}{设置具体时间}

\subsubsection{脚本同步}

\subsubsubsection{编写脚本}

在其他虚拟机上写一个脚本与虚拟机1的时间同步,设置为每10分钟同步一次时间。

\begin{lstlisting}[language=bash]
    crontab -e
\end{lstlisting}

输入以下内容。

\begin{lstlisting}[language=bash]
    0-59/10 * * * * /usr/sbin/ntpdate lipengda001
\end{lstlisting}

\subsubsubsection{同步}

输入命令立刻进行一次时间同步。

\begin{lstlisting}[language=bash]x
    /usr/sbin/ntpdate lipengda001
\end{lstlisting}

\notebox{
    如果同步失败,可以尝试关闭 \texttt{lipengda001} 虚拟机的防火墙。 \\
    \texttt{service firewalld stop}
}

\img{6.2.2.1}{时间同步}

\subsection{Hadoop 集群模式部署}

\subsubsection{主节点部署}

\subsubsubsection{重新解压}

进入app目录下,重新解压一个hadoop文件到其他文件夹。

\begin{lstlisting}[language=bash]
    cd ~/app
    tar -zxvf hadoop-2.5.0.tar.gz -C /tmp
\end{lstlisting}

了与之前的hadoop伪分布式模式区分,将其重命名为 \texttt{hadoop}。

\begin{lstlisting}[language=bash]
    mv /tmp/hadoop-2.5.0 ~/app/hadoop
\end{lstlisting}

\img{7.1.1.1}{重新解压}

\subsubsubsection{修改配置文件}

进入\texttt{hadoop/etc/hadoop}目录。

\begin{lstlisting}[language=bash]
    cd ~/app/hadoop/etc/hadoop
\end{lstlisting}

\begin{enum}
    \item 配置 \texttt{hadoop-env.sh}
    
    \begin{lstlisting}[language=bash]
    vi hadoop-env.sh
    \end{lstlisting}
    
    修改 \texttt{JAVA\_HOME} 为 \texttt{/root/app/jdk1.8.0\_131}。

    \img{7.1.2.1}{配置\texttt{hadoop-env.sh}}

    \item 配置 \texttt{core-site.xml}
    
    \begin{lstlisting}[language=bash]
    vi core-site.xml
    \end{lstlisting}

    添加以下内容。

    \begin{lstlisting}[language=xml]
    <configuration>
        <property>
            <name>fs.default.name</name>
            <value>hdfs://lipengda001:9000</value>
        </property>

        <property>
            <name>hadoop.tmp.dir</name>
            <value>/root/app/hadoop/data/tmp</value>
        </property>

        <property>
            <name>fs.trash.interval</name>
            <value>420</value>
        </property>
    </configuration>
    \end{lstlisting}

    \img{7.1.2.2}{配置\texttt{core-site.xml}}

    并在 \texttt{/root/app/hadoop} 目录下创建 \texttt{data/tmp} 目录。

    \begin{lstlisting}[language=bash]
    mkdir -p /root/app/hadoop/data/tmp
    \end{lstlisting}

    \item 配置 \texttt{hdfs-site.xml}
    
    \begin{lstlisting}[language=bash]
    vi hdfs-site.xml
    \end{lstlisting}

    添加以下内容。

    \begin{lstlisting}[language=xml]
    <configuration>
        <property>
            <name>dfs.namenode.secondary.http-address</name>
            <value>lipengda003:50090</value>
        </property>
    </configuration>
    \end{lstlisting}

    \img{7.1.2.3}{配置\texttt{hdfs-site.xml}}

    \item 配置 \texttt{mapred-site.xml}
    
    \begin{lstlisting}[language=bash]
    mv mapred-site.xml.template mapred-site.xml
    vi mapred-site.xml
    \end{lstlisting}

    添加以下内容。

    \begin{lstlisting}[language=xml]
    <configuration>
        <property>
            <name>mapreduce.framework.name</name>
            <value>yarn</value>
        </property>

        <property>
            <name>mapreduce.jobhistory.address</name>
            <value>lipengda001:10020</value>
        </property>

        <property>
            <name>mapreduce.jobhistory.webapp.address</name>
            <value>lipengda001:19888</value>
        </property>
    </configuration>
    \end{lstlisting}

    \img{7.1.2.4}{配置\texttt{mapred-site.xml}}

    \item 配置 \texttt{yarn-site.xml}
    
    \begin{lstlisting}[language=bash]
    vi yarn-site.xml
    \end{lstlisting}

    添加以下内容。

    \begin{lstlisting}[language=xml]
    <configuration>
        <property>
            <name>yarn.resourcemanager.hostname</name>
            <value>lipengda002</value>
        </property>

        <property>
            <name>yarn.nodemanager.aux-services</name>
            <value>mapreduce_shuffle</value>
        </property>

        <property>
            <name>yarn.log-aggregation-enable</name>
            <value>true</value>
        </property>

        <property>
            <name>yarn.log-aggregation.retain-seconds</name>
            <value>420</value>
        </property>
    </configuration>
    \end{lstlisting}

    \img{7.1.2.5}{配置\texttt{yarn-site.xml}}

    \item 配置 \texttt{slaves}
    
    \begin{lstlisting}[language=bash]
    vi slaves
    \end{lstlisting}

    添加以下内容。

    \begin{lstlisting}[language=bash]
    lipengda001
    lipengda002
    lipengda003
    \end{lstlisting}

    \img{7.1.2.6}{配置\texttt{slaves}}

    \subsubsection{集群节点分发与启动}

    \subsubsubsection{修改\texttt{/etc/profile}文件与分发}

    \begin{lstlisting}[language=bash]
    vi /etc/profile
    \end{lstlisting}

    注释掉 \texttt{HADOOP\_HOME} 和 相关\texttt{PATH} 的配置。

    \img{7.1.2.7}{修改\texttt{/etc/profile}文件}

    \begin{lstlisting}[language=bash]
    source /etc/profile
    \end{lstlisting}

    同步到其他虚拟机。

    \begin{lstlisting}[language=bash]
    scp -r /etc/profile lipengda002:/etc/profile
    scp -r /etc/profile lipengda003:/etc/profile
    \end{lstlisting}

    \img{7.1.2.8}{同步\texttt{/etc/profile}文件}

    在其他虚拟机上执行 \texttt{source /etc/profile}。

    将hadoop发送到其他两台虚拟机。

    为方便传输,先将文件夹压缩。

    \begin{lstlisting}[language=bash]
    tar -zcvf hadoop.tar.gz hadoop/
    \end{lstlisting}

    \begin{lstlisting}[language=bash]
    scp -r ~/app/hadoop.tar.gz lipengda002:~/app
    scp -r ~/app/hadoop.tar.gz lipengda003:~/app
    \end{lstlisting}

    \img{7.1.2.9}{发送hadoop文件}

    然后在其他两台虚拟机上解压。

    \begin{lstlisting}[language=bash]
    tar -zxvf hadoop.tar.gz -C .
    \end{lstlisting}
\end{enum}

\subsubsubsection{格式化}

\begin{lstlisting}[language=bash]
    bin/hdfs namenode -format
\end{lstlisting}

出现以下内容说明格式化成功。

\img{7.2.2.1}{格式化}

\subsubsubsection{启动}

\begin{enum}
    \item 在 \texttt{lipengda001} 虚拟机上启动。
    
    \begin{lstlisting}[language=bash]
    sbin/start-dfs.sh
    \end{lstlisting}

    在三台虚拟机上分别输入 \texttt{jps} 查看进程。

    \img{7.2.3.1}{启动(1)(\texttt{lipengda001})}
    \img{7.2.3.2}{启动(1)(\texttt{lipengda002})}
    \img{7.2.3.3}{启动(1)(\texttt{lipengda003})}

    \item 在 \texttt{lipengda002} 虚拟机上启动 \texttt{yarn}。
    
    \begin{lstlisting}[language=bash]
    sbin/start-yarn.sh
    \end{lstlisting}

    在三台虚拟机上分别输入 \texttt{jps} 查看进程。

    \img{7.2.3.4}{启动(2)(\texttt{lipengda002})}
    \img{7.2.3.5}{启动(2)(\texttt{lipengda001})}
    \img{7.2.3.6}{启动(2)(\texttt{lipengda003})}

    
    \item 在 \texttt{lipengda001} 虚拟机上启动 historyserver。
    
    \begin{lstlisting}[language=bash]
    sbin/mr-jobhistory-daemon.sh start historyserver
    \end{lstlisting}

    在三台虚拟机上分别输入 \texttt{jps} 查看进程。

    \img{7.2.3.7}{启动(3)(\texttt{lipengda001})}
    \img{7.2.3.8}{启动(3)(\texttt{lipengda002})}
    \img{7.2.3.9}{启动(3)(\texttt{lipengda003})}
\end{enum}

\subsection{分布式离线计算框架---MapReduce}

\subsubsection{安装Linux版本的eclipse}

\notebox{
    由于我安装的是最小安装的CentOS,没有安装图形化界面,因此无法安装eclipse。跳过此步骤。
}

\subsubsection{MapReduce实例——单词统计}

\begin{enum}
\item 创建一个Java项目TestHadoop

\begin{lstlisting}[language=bash]
    mkdir ~/app/workspace/TestHadoop
    cd ~/app/workspace/TestHadoop
\end{lstlisting}

\item 导入jar包

创建一个 \texttt{libs} 文件夹.

\begin{lstlisting}[language=bash]
    mkdir ~/libs
\end{lstlisting}

将 Hadoop的 \texttt{jar} 包拷贝到 \texttt{libs} 文件夹。

\begin{lstlisting}[language=bash]
    cp ~/app/hadoop/share/hadoop/common/*.jar ~/libs
    cp ~/app/hadoop/share/hadoop/hdfs/*.jar ~/libs
    cp ~/app/hadoop/share/hadoop/mapreduce/*.jar ~/libs
    cp ~/app/hadoop/share/hadoop/yarn/*.jar ~/libs
    cp ~/app/hadoop/share/hadoop/common/lib/*.jar ~/libs
    cp ~/app/hadoop/share/hadoop/hdfs/lib/*.jar ~/libs
    cp ~/app/hadoop/share/hadoop/mapreduce/lib/*.jar ~/libs
    cp ~/app/hadoop/share/hadoop/yarn/lib/*.jar ~/libs
\end{lstlisting}

\item 创建一个文本文件word.txt做测试文件

\begin{lstlisting}[language=bash]
    vi word.txt
\end{lstlisting}

输入以下内容。

\begin{lstlisting}[language=bash]
    hello java
    java hadoop
    spark hbase
    hello hadoop
    hello word
\end{lstlisting}

\item 创建包 \texttt{com.hadoop.mapreduce}

\begin{lstlisting}[language=bash]
    mkdir -p src/com/hadoop/mapreduce
\end{lstlisting}

\item 创建 \texttt{WordCountMapper.java} 文件

\begin{lstlisting}[language=bash]
    cd src/com/hadoop/mapreduce
    vi WordCountMapper.java
\end{lstlisting}

输入以下内容。

\begin{lstlisting}[language=java]
    package com.hadoop.mapreduce;

    import java.io.IOException;
    import org.apache.hadoop.io.LongWritable;
    import org.apache.hadoop.io.NullWritable;
    import org.apache.hadoop.io.Text;
    import org.apache.hadoop.mapreduce.Mapper;

    public class WordCountMapper extends Mapper<LongWritable, Text, NullWritable, LongWritable> {
        @Override
        protected void map(LongWritable key, Text value,Mapper<LongWritable, Text, NullWritable, LongWritable>.Context context) throws IOException, InterruptedException {
            String line = value.toString();
            //用空格进行分割
            String words[] = line.split(" ");
            context.write(NullWritable.get(), new LongWritable(words.length));
        }
    }
\end{lstlisting}

\item 创建 \texttt{WordCountReduce.java} 文件

\begin{lstlisting}[language=bash]
    vi WordCountReduce.java
\end{lstlisting}

输入以下内容。

\begin{lstlisting}[language=java]
    package com.hadoop.mapreduce;

    import java.io.IOException;
    import org.apache.hadoop.io.LongWritable;
    import org.apache.hadoop.io.NullWritable;
    import org.apache.hadoop.mapreduce.Reducer;

    public class WordCountReduce extends Reducer<NullWritable, LongWritable, NullWritable, LongWritable> {
        //数组分组合并输出
        @Override
        protected void reduce(NullWritable key, Iterable<LongWritable> v2s, Reducer<NullWritable, LongWritable, NullWritable, LongWritable>.Context context) throws IOException, InterruptedException {
            long counter = 0;
            for(LongWritable v:v2s){
                counter += v.get();
            }
            context.write(NullWritable.get(), new LongWritable(counter));
        }
    }
\end{lstlisting}

\item 创建 \texttt{WordCount.java} 文件

\begin{lstlisting}[language=bash]
    vi WordCount.java
\end{lstlisting}

输入以下内容。

\begin{lstlisting}[language=java]
    package com.hadoop.mapreduce;

    import org.apache.hadoop.conf.Configuration;
    import org.apache.hadoop.fs.Path;
    import org.apache.hadoop.io.LongWritable;
    import org.apache.hadoop.io.NullWritable;
    import org.apache.hadoop.mapreduce.Job;
    import org.apache.hadoop.mapreduce.lib.input.FileInputFormat;
    import org.apache.hadoop.mapreduce.lib.output.FileOutputFormat;

    public class WordCount {
        
        public static void main(String[] args) throws Exception{
            Configuration conf = new Configuration();
            Job job = Job.getInstance(conf);
            job.setJarByClass(WordCount.class);
            // Mapper方法名
            job.setMapperClass(WordCountMapper.class);
            // Reducer方法名
            job.setReducerClass(WordCountReduce.class);
            // Map输出的key类型
            job.setMapOutputKeyClass(NullWritable.class);
            // Map输出的value类型
            job.setMapOutputValueClass(LongWritable.class);
            // Reduce输出的key类型
            job.setOutputKeyClass(NullWritable.class);
            // Reduce 输出的value类型
            job.setOutputValueClass(LongWritable.class);
            // 读取的文件位置
            FileInputFormat.setInputPaths(job, new Path("file:///root/app/workspace/TestHadoop/word.txt"));
            // 处理完之后的数据存放位置,注意输出的文件夹如果已经存在会报错
            FileOutputFormat.setOutputPath(job, new Path("file:///root/app/workspace/TestHadoop/out"));
            job.waitForCompletion(true);
        }
    }
\end{lstlisting}

\item 编译运行

\begin{lstlisting}[language=bash]
    cd ~/app/workspace/TestHadoop
    javac -cp ~/libs/*: src/com/hadoop/mapreduce/*.java -d .
    java -cp ~/libs/*: com.hadoop.mapreduce.WordCount
\end{lstlisting}

\item 查看结果

\begin{lstlisting}[language=bash]
    cd out
    ls
    cat part-r-00000
\end{lstlisting}

\img{8.2.1}{查看结果}

\end{enum}

\subsubsection{MapReduce实例——温度统计}

根据MapReduce的原理,结合单词统计案例,实现温度统计。
需求:找出每年每月的2个最低温度时刻并进行升序排列。

\begin{enum}
    \item 创建一个Java项目TemperatureStats
    
    \begin{lstlisting}[language=bash]
    mkdir ~/app/workspace/TemperatureStats
    cd ~/app/workspace/TemperatureStats
    \end{lstlisting}
    
    \item 创建一个文本文件temperature.txt做测试文件
    
    \begin{lstlisting}[language=bash]
    vi temperature.txt
    \end{lstlisting}
    
    输入以下内容。
    
    \begin{lstlisting}[language=bash]
    2017-10-05 12:15:30     26
    2017-11-12 09:36:54     23
    2017-11-16 15:12:12     29
    2017-11-17 10:30:59     30
    2017-11-23 11:23:15     19
    2018-05-16 18:23:23     28
    2018-05-21 12:56:30     33
    2018-06-03 08:16:15     26
    2018-10-15 16:15:20     25
    2019-04-09 21:25:26     18
    2019-07-16 13:15:16     34
    2019-07-22 22:16:56     16
    2019-07-23 05:26:11     11
    \end{lstlisting}
    
    \item 创建包 \texttt{com.hadoop.temperature}
    
    \begin{lstlisting}[language=bash]
    mkdir -p src/com/hadoop/temperature
    \end{lstlisting}
    
    \item 创建 \texttt{TemperatureMapper.java} 文件
    
    \begin{lstlisting}[language=bash]
    cd src/com/hadoop/temperature
    vi TemperatureMapper.java
    \end{lstlisting}
    
    输入以下内容。
    
    \begin{lstlisting}[language=java]
    package com.hadoop.temperature;

    import java.io.IOException;
    import org.apache.hadoop.io.Text;
    import org.apache.hadoop.mapreduce.Mapper;

    public class TemperatureMapper extends Mapper<Object, Text, Text, Text> {
        @Override
        protected void map(Object key, Text value, Context context) throws IOException, InterruptedException {
            String line = value.toString();
            String[] parts = line.split("\\s+");
            if (parts.length == 3) {
                String date = parts[0];
                String time = parts[1];
                String temperature = parts[2];

                String yearMonth = date.substring(0, 7);
                String fullRecord = date + " " + time + " " + temperature;

                Text outputKey = new Text();
                Text outputValue = new Text();

                outputKey.set(yearMonth);
                outputValue.set(fullRecord);

                context.write(outputKey, outputValue);
            }
        }
    }
    \end{lstlisting}
    
    \item 创建 \texttt{TemperatureReducer.java} 文件
    
    \begin{lstlisting}[language=bash]
    vi TemperatureReducer.java
    \end{lstlisting}
    
    输入以下内容。
    
    \begin{lstlisting}[language=java]
    package com.hadoop.temperature;

    import java.io.IOException;
    import java.util.ArrayList;
    import java.util.Collections;
    import java.util.Comparator;
    import java.util.PriorityQueue;
    import org.apache.hadoop.io.Text;
    import org.apache.hadoop.mapreduce.Reducer;

    public class TemperatureReducer extends Reducer<Text, Text, Text, Text> {
        @Override
        protected void reduce(Text key, Iterable<Text> values, Context context) throws IOException, InterruptedException {
            ArrayList<String> records = new ArrayList<>();

            for (Text value : values) {
                records.add(value.toString());
            }

            Collections.sort(records, new Comparator<String>() {
                @Override
                public int compare(String o1, String o2) {
                    int temp1 = Integer.parseInt(o1.split("\\s+")[2]);
                    int temp2 = Integer.parseInt(o2.split("\\s+")[2]);
                    return Integer.compare(temp1, temp2);
                }
            });

            Text outputValue = new Text();

            int limit = Math.min(2, records.size());
            for (int i = 0; i < limit; i++) {
                outputValue.set(records.get(i));
                context.write(null, outputValue);
            }
        }
    }
    \end{lstlisting}
    
    \item 创建 \texttt{TemperatureStats.java} 文件
    
    \begin{lstlisting}[language=bash]
    vi TemperatureStats.java
    \end{lstlisting}
    
    输入以下内容。
    
    \begin{lstlisting}[language=java]
    package com.hadoop.temperature;

    import org.apache.hadoop.conf.Configuration;
    import org.apache.hadoop.fs.Path;
    import org.apache.hadoop.io.Text;
    import org.apache.hadoop.mapreduce.Job;
    import org.apache.hadoop.mapreduce.lib.input.FileInputFormat;
    import org.apache.hadoop.mapreduce.lib.output.FileOutputFormat;

    public class TemperatureStats {
        public static void main(String[] args) throws Exception {
            Configuration conf = new Configuration();
            Job job = Job.getInstance(conf, "Temperature Stats");
            job.setJarByClass(TemperatureStats.class);
            job.setMapperClass(TemperatureMapper.class);
            job.setReducerClass(TemperatureReducer.class);
            job.setOutputKeyClass(Text.class);
            job.setOutputValueClass(Text.class);

            FileInputFormat.addInputPath(job, new Path("file:///root/app/workspace/TemperatureStats/temperature.txt"));
            FileOutputFormat.setOutputPath(job, new Path("file:///root/app/workspace/TemperatureStats/out"));

            System.exit(job.waitForCompletion(true) ? 0 : 1);
        }
    }
    \end{lstlisting}
    
    \item 编译运行
    
    \begin{lstlisting}[language=bash]
    cd ~/app/workspace/TemperatureStats
    javac -cp ~/libs/*: src/com/hadoop/temperature/*.java -d .
    java -cp ~/libs/*: com.hadoop.temperature.TemperatureStats
    \end{lstlisting}
    
    \item 查看结果
    
    \begin{lstlisting}[language=bash]
    cd out
    ls
    cat part-r-00000
    \end{lstlisting}
    
    \img{8.3.1}{查看结果}
    
    \end{enum}
    

\section{实验结果}

\subsection{成功安装虚拟机,并于Windows通信}
\img{0.1.2.4.4}{Windows Ping 虚拟机}

\subsection{成功运行Hadoop单例模式}
\img{0.2.3.4.2}{查看Hadoop单例模式运行结果}

\subsection{成功运行Hadoop伪分布式模式}
\img{0.3.2.2.1}{启动Hadoop伪分布式模式}

\subsection{成功克隆虚拟机并与Windows通信}
\img{4.2.4}{Windows ping 三个虚拟机}

\subsection{成功运行Hadoop集群模式}
\img{7.2.3.7}{集群模式(\texttt{lipengda001})}
\img{7.2.3.8}{集群模式(\texttt{lipengda002})}
\img{7.2.3.9}{集群模式(\texttt{lipengda003})}

\subsection{成功使用MapReduce框架进行单词统计}
\img{8.2.1}{MapReduce单词统计结果}

\subsection{成功使用MapReduce框架进行温度统计}
\img{8.3.1}{MapReduce温度统计结果}

\section{实验总结}

在本次实验中,我完成了Linux虚拟机的配置、Hadoop的单例、伪分布式和集群模式的搭建,以及MapReduce框架的使用。在实验过程中,我学会了如何配置Linux虚拟机网络、ssh与时钟同步,学习了搭建Hadoop集群,以及使用MapReduce框架进行了单词统计。通过本次实验,我对Hadoop有了更深入的了解,也对云计算有了更多的认识。

在实验中,我遇到的问题及解决方法、一些与实验手册不同的地方和我自己的思考,我记录在了蓝色“NOTE”框中。

\end{document}