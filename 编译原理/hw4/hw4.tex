\documentclass[a4paper, body={18cm,22cm}]{article}
\usepackage{ctex}
\usepackage{amsmath}
\usepackage{forest}


\title{Homework 4: 上下文无关文法}
\author{李鹏达 10225101460}
\date{}
\begin{document}
\maketitle

\subsection*{1.为下列语言设计上下文无关文法。 也请思考下列语言可不可以设计正规表达式? }

\begin{enumerate}
    \item[a)] 满足这样条件的二进制串:每个 0 之后都紧跟着至少一个 1;  
    
    $S \to 1S \mid 01S \mid \epsilon
    $

    写成正规表达式:$(1^*011^*)^*$
    \item[b)] 0 和 1 个数相等的二进制串;
    
    $S \to 0S1 \mid 1S0 \mid SS \mid \epsilon$

    无法写成正规表达式。
    \item[c)] 不含 011 子串的二进制串;  
    
    $
    S \to 0A \mid 1S \mid \epsilon \\
    A \to 0A \mid 1B \mid \epsilon \\
    B \to 0A \mid \epsilon
    $

    写成正规表达式:$1^*|(1^*0(0|10)^*)|(1^*0(0|10)^*1)$
    \item[d)] 具有形式$xy$的二进制串,$x \ne y$; 
    
    $S \to 1S1 \mid 0S0 \mid A \\
    A \to 1B0 \mid 0B1 \\
    B \to 1B \mid 0B \mid \epsilon
    $
    无法写成正规表达式。
    \item[e)] 形如$xx$的二进制串;
    
    $S \to 0S0 \mid 1S1 \mid \epsilon$

    无法写成正规表达式。
\end{enumerate}

\subsection*{2.考虑文法}

$
\begin{aligned}
    &S \to (L)\mid a \\
    &L \to L,S\mid S
\end{aligned}
$


\begin{enumerate}
    \item[a)] 列出终结符、非终结符和开始符号; 
    
    终结符:``$a$'',``,'', ``('', ``)'';非终结符:$S, L$;开始符号:$S$
    \item[b)] 给出下列句子的语法树 
    
    i)(a, a)  

    ii) (a, (a, a))   
    
    iii) (a, ((a, a), (a, a))) 

    \subsubsection*{i) $(a, a)$}
    \begin{center}
    \begin{tikzpicture}[level distance=10mm, sibling distance=15mm]
    \node {$S$}
        child {node {$(L)$}
            child {node {$($}}
            child {node {$L$}
                child {node {$L$}
                    child {node {$S$}
                        child {node {$a$}}
                    }
                }
                child {node {$,$}}
                child {node {$S$}
                    child {node {$a$}}
                }
            }
            child {node {$)$}}
        };
    \end{tikzpicture}
    \end{center}
    
    \subsubsection*{ii) $(a, (a, a))$}
    \begin{center}
    \begin{tikzpicture}[level distance=10mm, sibling distance=15mm]
    \node {$S$}
        child {node {$(L)$}
            child {node {$($}}
            child {node {$L$}
                child {node {$L$}
                    child {node {$S$}
                        child {node {$a$}}
                    }
                }
                child {node {$,$}}
                child {node {$S$}
                    child {node {$(L)$}
                        child {node {$($}}
                        child {node {$L$}
                            child {node {$L$}
                                child {node {$S$}
                                    child {node {$a$}}
                                }
                            }
                            child {node {$,$}}
                            child {node {$S$}
                                child {node {$a$}}
                            }
                        }
                        child {node {$)$}}
                    }
                }
            }
            child {node {$)$}}
        };
    \end{tikzpicture}
    \end{center}
    
    \subsubsection*{iii) $(a, ((a, a), (a, a)))$}
    \begin{center}
        \begin{tikzpicture}[level distance=15mm, sibling distance=15mm]
            \node {$S$}
                child {node {$(L)$}
                    child {node {$($}}
                    child {node {$L$}
                        child {node {$L$}
                            child {node {$S$}
                                child {node {$a$}}
                            }
                        }
                        child {node {$,$}}
                        child {node {$S$}
                            child {node {$(L)$}
                                child {node {$($}}
                                child {node {$L$}
                                    [sibling distance=15mm] % 调整该层的间距
                                    child {node {$L$}
                                        child {node {$S$}
                                            child {node {$(L)$}
                                                child {node {$($}}
                                                child {node {$L$}
                                                    [sibling distance=10mm] % 增大间距
                                                    child {node {$L$}
                                                        child {node {$S$}
                                                            child {node {$a$}}
                                                        }
                                                    }
                                                    child {node {$,$}}
                                                    child {node {$S$}
                                                        child {node {$a$}}
                                                    }
                                                }
                                                child {node {$)$}}
                                            }
                                        }
                                    }
                                    child {node {$,$}}
                                    child {node {$S$}
                                        child {node {$(L)$}
                                            child {node {$($}}
                                            child {node {$L$}
                                                [sibling distance=10mm]
                                                child {node {$L$}
                                                    child {node {$S$}
                                                        child {node {$a$}}
                                                    }
                                                }
                                                child {node {$,$}}
                                                child {node {$S$}
                                                    child {node {$a$}}
                                                }
                                            }
                                            child {node {$)$}}
                                        }
                                    }
                                }
                                child {node {$)$}}
                            }
                        }
                    }
                    child {node {$)$}}
                };
        \end{tikzpicture}
    \end{center}
    

    \item[c)] 构造 b)中句子的最左推导; 
    
    \subsubsection*{i) $(a, a)$}
\begin{align*}
S &\Rightarrow (L) \\
&\Rightarrow (L,S) \\
&\Rightarrow (S,S) \\
&\Rightarrow (a,S) \\
&\Rightarrow (a,a)
\end{align*}

\subsubsection*{ii) $(a, (a, a))$}
\begin{align*}
S &\Rightarrow (L) \\
&\Rightarrow (L,S) \\
&\Rightarrow (S,S) \\
&\Rightarrow (a,S) \\
&\Rightarrow (a,(L)) \\
&\Rightarrow (a,(L,S)) \\
&\Rightarrow (a,(S,S)) \\
&\Rightarrow (a,(a,S)) \\
&\Rightarrow (a,(a,a))
\end{align*}

\subsubsection*{iii) $(a, ((a, a), (a, a)))$}
\begin{align*}
S &\Rightarrow (L) \\
&\Rightarrow (L,S) \\
&\Rightarrow (S,S) \\
&\Rightarrow (a,S) \\
&\Rightarrow (a,(L)) \\
&\Rightarrow (a,(L,S)) \\
&\Rightarrow (a,(S,S)) \\
&\Rightarrow (a,((L),S)) \\
&\Rightarrow (a,((L,S),S)) \\
&\Rightarrow (a,((S,S),S)) \\
&\Rightarrow (a,((a,S),S)) \\
&\Rightarrow (a,((a,a),S)) \\
&\Rightarrow (a,((a,a),(L))) \\
&\Rightarrow (a,((a,a),(L,S))) \\
&\Rightarrow (a,((a,a),(S,S))) \\
&\Rightarrow (a,((a,a),(a,S))) \\
&\Rightarrow (a,((a,a),(a,a)))
\end{align*}
    \item[d)] 构造 b)中句子的最右推导; 
    
    \subsubsection*{i) $(a, a)$}
\begin{align*}
S &\Rightarrow (L) \\
&\Rightarrow (L,S) \\
&\Rightarrow (L,a) \\
&\Rightarrow (S,a) \\
&\Rightarrow (a,a)
\end{align*}

\subsubsection*{ii) $(a, (a, a))$}
\begin{align*}
S &\Rightarrow (L) \\
&\Rightarrow (L,S) \\
&\Rightarrow (L,(L)) \\
&\Rightarrow (L,(L,S)) \\
&\Rightarrow (L,(L,a)) \\
&\Rightarrow (L,(S,a)) \\
&\Rightarrow (L,(a,a)) \\
&\Rightarrow (S,(a,a)) \\
&\Rightarrow (a,(a,a))
\end{align*}

\subsubsection*{iii) $(a, ((a, a), (a, a)))$}
\begin{align*}
S &\Rightarrow (L) \\
&\Rightarrow (L,S) \\
&\Rightarrow (L,(L)) \\
&\Rightarrow (L,(L,S)) \\
&\Rightarrow (L,(L,(L))) \\
&\Rightarrow (L,(L,(L,S))) \\
&\Rightarrow (L,(L,(L,a))) \\
&\Rightarrow (L,(L,(S,a))) \\
&\Rightarrow (L,(L,(a,a))) \\
&\Rightarrow (L,(S,(a,a))) \\
&\Rightarrow (L,((L),(a,a))) \\
&\Rightarrow (L,((L,S),(a,a))) \\
&\Rightarrow (L,((L,a),(a,a))) \\
&\Rightarrow (L,((S,a),(a,a))) \\
&\Rightarrow (L,((a,a),(a,a))) \\
&\Rightarrow (S,((a,a),(a,a))) \\
&\Rightarrow (a,((a,a),(a,a)))
\end{align*}
    \item[e)] 该文法产生的语言是什么? (可以用自然语言描述出来,也可以用集合的
    形式表示出来)

    可以嵌套的的二元组或$a$,其中二元组形式为$(x_1,x_2)$,其中$x_1$和$x_2$可以是二元组或$a$。
\end{enumerate}

\subsection*{3. 考虑文法 }

$S \to aSbS \mid bSaS \mid \epsilon$
 
为某个句子构造两个不同的最左推导,以证明它是二义性的。
 
句子:\( abab \)

1. \( S \Rightarrow aSbS \Rightarrow abS \Rightarrow abaSbS \Rightarrow ababS \Rightarrow abab \)

2. \( S \Rightarrow aSbS \Rightarrow abSaSbS \Rightarrow abaSbS \Rightarrow ababS \Rightarrow abab \)

存在 2 种不同的最左推导,存在二义性。


\newpage

\subsection*{1.考虑文法}

\begin{align*}
S &\rightarrow (L) \mid a \\
L &\rightarrow L, S \mid S
\end{align*}

消除左递归。

\begin{align*}
L &\to SL'\\
L' &\to ,SL' \mid \epsilon
\end{align*}

消除左递归后的文法:

\begin{align*}
S &\rightarrow (L) \mid a \\
L &\rightarrow SL' \\
L' &\rightarrow ,SL' \mid \epsilon
\end{align*}

\subsection*{2.下面文法 G[S]}

\begin{align*}
S &\rightarrow AbB \mid A \\
A &\rightarrow AB \mid caB \mid B \\
B &\rightarrow Aa \mid b
\end{align*}

消除左递归。

将 (3) 式带入,得到:


\begin{align*}
S &\to AbAs \mid Abb \mid A \\
A &\to AAa \mid Ab \mid caAa \mid cab \mid Aa \mid b \\
\end{align*}

\begin{align*}
A &\to caAaA' \mid cabA' \mid bA' \\
A' &\to AaA' \mid bA' \mid aA' \mid \epsilon 
\end{align*}

消除后的文法:

\begin{align*}
S &\to AbAs \mid Abb \mid A \\
A &\to caAaA' \mid cabA' \mid bA' \\
A' &\to AaA' \mid bA' \mid aA' \mid \epsilon
\end{align*}

\subsection*{3.下面文法 G[S]}

\begin{align*}
S &\rightarrow aFbM \mid F \\
F &\rightarrow M \mid abc \\
M &\rightarrow abF \mid c
\end{align*}

消除形如 $A \rightarrow B$ 这种单产生式,并进行文法处理(左递归和公共左因子)。

(3)式带入(2)式,得到:
\begin{align*}
F &\to abF \mid c \mid abc \\
\end{align*}

提取公共左因子,得到:
\begin{align*}
F &\to abF' \mid c \\
F' &\to F \mid c \\
\end{align*}

带入$F$,得到
\begin{align*}
F' &\to abF' \mid c \\
\end{align*}

发现$F$和$F'$是相同的,用$F$替换$F'$,得到:
\begin{align*}
F &\to abF \mid c \\
\end{align*}

由于
$$
M \to abF \mid c
$$

$F$ 和$M$是相同的,用$F$替换$M$,得到:
\begin{align*}
S \to aFbF \mid F
\end{align*}

将$F$带入,得到
\begin{align*}
S \to aFbM \mid abF \mid c \\
\end{align*}

提取公共左因子,得到:
\begin{align*}
S &\to aS' \mid c \\
S' &\to FbF \mid bF \\
\end{align*}

文法最终化为
\begin{align*}
S &\to aS' \mid c \\
S' &\to FbF \mid bF \\
F &\to abF \mid c \\
\end{align*}

\end{document}
