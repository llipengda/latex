\documentclass[a4paper, body={18cm,22cm}]{article}
\usepackage{ctex}
\usepackage{amsmath}


\title{Homework1: 正则表达式课后练习}
\author{李鹏达 10225101460}
\date{}

\begin{document}
\maketitle

\subsection*{1. 描述下列正则式(正则)定义的语言(请用汉语直白地描述该语言)}

\begin{enumerate}
    \item[(a)] \(0(0|1)^*0\)

    以0开始,以0结尾的01串。
    \item[(b)] \(((\varepsilon | 0)1^*)^*\)
    
    由0 和 1 组成的所有可能的字符串,包括空串。
    \item[(c)] \((0|1)^*0(0|1)(0|1)\)
    
    长度至少为3,倒数第三位是0的01串。
    \item[(d)] \(0^*10^*10^*10^*\)
    
    长度至少为3,有且只有3个1的01串。
    \item[(e)] \((00|11)^*((01|10)(00|11)^*(01|10)(00|11)^*)^*\)
    
    由偶数个0和偶数个1组成的01串。
\end{enumerate}

\subsection*{2. 为下列语言写出正规定义(或正则式、正规文法、有限自动机)}

\begin{enumerate}
    \item[(a)] 包含五个元音,且按顺序排列的所有字母串(\(RE\))
    
    $
    other \to (b|c|d|f|g|h|j|k|l|m|n|p|q|r|s|t|v|w|x|y|z) \\
    str \to {other}^*a(other|a)^*e(other|e)^*i(other|i)^*o(other|o)^*u(other|u)^*
    $
    \item[(b)] 字母按字典顺序排列的所有字母串(\(RE\))
    
    $a^*b^*c^*d^*e^*f^*g^*h^*i^*j^*k^*l^*m^*n^*o^*p^*q^*r^*s^*t^*u^*v^*w^*x^*y^*z^*$
    \item[(c)] 以 \texttt{/*} 开始,\texttt{*/} 结束的注释,中间不能包含 \texttt{*/},除非包含在 \texttt{"} 和 \texttt{"} 中(\(RE\))
    
    $\backslash/\backslash\!*\!([\verb|^|\!*\!\verb|"|]^*|\verb|"|.*?\verb|"||\verb|"||\backslash\!*^+\![\verb|^|\!*\!/])^*\backslash\!*\!\backslash/$
    \item[(d)] 不包含重复数字的数字串(正规定义)
    
    $\verb|^|(?!.^*(.).^*\verb|\|1)[0\!-\!9]^*\verb|$|$
    \item[(e)] 包含至少一个重复数字的数字串(正规定义)
    
    $\verb|^|(?!.^*(.).^*\verb|\|1\verb|\|1)[0\!-\!9]^*\verb|$|$
    \item[(f)] 包含偶数个 0 和奇数个 1 的 0、1 串(正规定义)

    $startsWithOne  \to 1(00|11)^*((01|10)(00|11)^*(01|10)(00|11)^*)^* \\
    startsWithZero \to 0(00|11)^*((01|10)(00|11)^*(01|10)(00|11)^*(01|10)(00|11)^*)^* \\
    str \to startsWithOne | startsWithZero
    $
\end{enumerate}

\end{document}
