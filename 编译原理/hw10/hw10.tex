\documentclass[a4paper, body={18cm,22cm}]{article}
\usepackage{ctex}
\usepackage{amsmath}
\usepackage{makecell}
\usepackage{xcolor}
\usepackage{listings}

\lstset{
    language = [x86masm]Assembler,
    xleftmargin = 3em,xrightmargin = 3em, aboveskip = 1em,
	backgroundcolor = \color{white}, % 背景色
	basicstyle = \small\ttfamily, % 基本样式 + 小号字体
	rulesepcolor= \color{gray}, % 代码块边框颜色
	breaklines = true, % 代码过长则换行
	numbers = left, % 行号在左侧显示
	numberstyle = \footnotesize, % 行号字体
    numbersep = 14pt, 
	keywordstyle = \color{blue!50!red!100}, % 关键字颜色
	commentstyle =\color{red!50!green!50!blue!60}, % 注释颜色
	stringstyle = \color{red}, % 字符串颜色
	frame = none, % 用(带影子效果)方框框住代码块
	showspaces = false, % 不显示空格
	columns = fixed, % 字间距固定
} 

\title{Homework 10: 中间代码生成}
\author{李鹏达 10225101460}
\date{}
\begin{document}
\maketitle

\subsection*{1. 下面是一段 C 程序,请写出对应生成的三地址代码。}

\begin{lstlisting}[language=C]
{
    int i = 0;
    int a[10];
    while (i <= 9) {
        a[i] = i;
        i++;
    }
}
\end{lstlisting}

\noindent\textbf{{\heiti 解答}}\\ 

\begin{lstlisting}
mov  i,,0
le   t1,i,9
jmpf t1,,10
mov  t2,,c    ; where c is base_a - 0 * 4
mult t3,i,4
mov  t2[t3],,i
add  t4,i,1
mov  i,,t4
jmp  ,,2
              ;
\end{lstlisting}

\end{document}
