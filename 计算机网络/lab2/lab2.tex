\documentclass{article}
\usepackage{fancyhdr}
\usepackage{ctex}
\usepackage{listings}
\usepackage{graphicx}
\usepackage[a4paper, body={18cm,22cm}]{geometry}
\usepackage{amsmath,amssymb,amstext,wasysym,enumerate,graphicx}
\usepackage{float,abstract,booktabs,indentfirst,amsmath}
\usepackage{array}
\usepackage{booktabs}
\usepackage{multirow}
\usepackage{url}
\usepackage{diagbox}
\renewcommand\arraystretch{1.4}
\usepackage{indentfirst}
\setlength{\parindent}{2em}
\usepackage{enumitem}
\setmonofont{Consolas}
\usepackage{listings}
\usepackage{xcolor}
\usepackage{makecell}
\setCJKmonofont{黑体}
\lstset{
    % language = C,
    xleftmargin = 3em,xrightmargin = 3em, aboveskip = 1em,
	backgroundcolor = \color{white}, % 背景色
	basicstyle = \small\ttfamily, % 基本样式 + 小号字体
	rulesepcolor= \color{gray}, % 代码块边框颜色
	breaklines = true, % 代码过长则换行
	numbers = left, % 行号在左侧显示
	numberstyle = \small, % 行号字体
    numbersep = -14pt, 
    keywordstyle=\color{purple}\bfseries, % 关键字颜色
    commentstyle =\color{red!50!green!50!blue!60}, % 注释颜色
    stringstyle = \color{red}, % 字符串颜色
    morekeywords={ASSERT, int64_t, uint32_t},
	frame = shadowbox, % 用(带影子效果)方框框住代码块
	showspaces = false, % 不显示空格
	columns = fixed, % 字间距固定
} 
\lstset{
    sensitive=true,
    moreemph={ASSERT, NULL}, emphstyle=\color{red}\bfseries,
    moreemph=[2]{int64_t, uint32_t, tid_t, uint8_t, int16_t, uint16_t, int32_t, size_t}, emphstyle=[2]\color{purple}\bfseries,
    }
%--------------------页眉--------------------%
\pagestyle{fancy}
\fancyhead[L]{}
\fancyhead[R]{}
\fancyhead[C]{华东师范大学软件工程学院实验报告}
\fancyfoot[C]{-\thepage-}
\renewcommand{\headrulewidth}{1.5pt}
%--------------------标题--------------------%
\begin{document}
\begin{center}
  \LARGE{{\textbf{\heiti 华东师范大学软件工程学院实验报告}}}
  \begin{table}[H]
    \centering
    \begin{tabular}{p{2cm}p{4cm}<{\centering}p{1cm}p{2cm}p{4cm}<{\centering}}
      实验课程:    & 计算机网络 & \quad & 年\qquad 级: & 2022级      \\ \cline{2-2} \cline{5-5}
      实验编号:    & Lab 02     & \quad & 实验名称:    & Ethernet
      \\ \cline{2-2} \cline{5-5}
      姓\qquad 名: & 李鹏达     & \quad & 学\qquad 号: & 10225101460 \\ \cline{2-2} \cline{5-5}
    \end{tabular}
  \end{table}
\end{center}
\rule{\textwidth}{1pt}
%--------------------正文--------------------%
\section{实验目的}
\begin{enumerate}[noitemsep, label={{\arabic*})}]
  \item 学会通过Wireshark获取以太网的帧
  \item 掌握以太网帧的结构
  \item 分析以太网地址范围
  \item 分析以太网的广播帧
\end{enumerate}
\section{实验内容与实验步骤}
\subsection{实验内容}


\subsubsection{获取以太网的帧}
在命令行中使用\texttt{ping}命令发起\texttt{ICMP}请求,然后使用\texttt{Wireshark}捕获\texttt{以太网}数据包。

\subsubsection{分析以太网的帧}

分析\texttt{以太网}的帧,画出帧结构。

\subsubsection{分析以太网的地址范围}

分析\texttt{以太网}的地址范围,画出图示关系图。

\subsubsection{分析以太网的广播帧}

启动Wireshark,在菜单栏的捕获\(\to \)选项中进行设置,选择已连接的以太网,设置捕获过滤器为\texttt{ether multicast},捕获\texttt{以太网}的广播帧。

分析\texttt{以太网}的广播帧,回答以下问题:

\begin{enumerate}[noitemsep]
  \item 以太网广播帧的地址是什么,以标准的形式写在Wireshark上显示?
  \item 哪几个比特位的以太网地址是用来确定是单播或多播/广播?
\end{enumerate}

\subsubsection{问题讨论}

\begin{enumerate}[noitemsep]
  \item How long are the combined IEEE 802.3 and LLC headers compared to the
        DIX Ethernet headers?
  \item How does the receiving computer know whether the frame is DIX Ethernet
        or IEEE 802.3?
  \item If IEEE 802.3 has no Type field, then how is the next higher layer
        determined?
\end{enumerate}


\subsection{实验步骤}

\begin{enumerate}[noitemsep, label={{\arabic*})}]
  \item 打开命令行,使用\texttt{ping}命令发起\texttt{ICMP}请求

        \begin{lstlisting}
    PS> ping www.baidu.com
  \end{lstlisting}

  \item 启动\texttt{Wireshark},在菜单栏的捕获\(\to \)选项中进行设置,选择已连接的以太网,设置捕获过滤器为\texttt{icmp},将混杂模式设为关闭,勾选
        \texttt{enable MAC name resolution}.然后开始捕获。
  \item 回到命令行,再次使用\texttt{ping}命令发起\texttt{ICMP}请求
        \begin{lstlisting}
    PS> ping www.baidu.com
  \end{lstlisting}
  \item 回到\texttt{Wireshark},停止捕获。
  \item 分析捕获到的\texttt{以太网}的帧,画出帧结构。
  \item 分析以太网的地址范围,画出图示关系图。
  \item 启动\texttt{Wireshark},在菜单栏的捕获\(\to \)选项中进行设置,选择已连接的以太网,设置捕获过滤器为\texttt{ether multicast},捕获\texttt{以太网}的广播帧。
  \item 问题讨论
\end{enumerate}

\section{实验环境}


\begin{itemize}[noitemsep]
  \item 操作系统:\texttt{Windows 11 家庭中文版 23H2 22631.2715}
  \item 网络适配器:\texttt{Killer(R) Wi-Fi 6 AX1650i 160MHz Wireless Network \\ Adapter(201NGW)}
  \item \texttt{Wireshark}:\texttt{Version 4.2.0 (v4.2.0-0-g54eedfc63953)}
  \item \texttt{wget}:\texttt{GNU Wget 1.21.4 built on mingw32}
\end{itemize}


\section{实验过程与分析}

\subsection{获取以太网的帧}

首先,我们在命令行中使用\texttt{ping}命令发起\texttt{ICMP}请求。

\begin{figure}[H]
  \centering
  \includegraphics[width=0.8\textwidth]{images/01.png}
  \caption{使用\texttt{ping}命令发起\texttt{ICMP}请求}
\end{figure}

打开\texttt{Wireshark},在菜单栏的捕获\(\to \)选项中进行设置,选择已连接的以太网,设置捕获过滤器为\texttt{icmp},将混杂模式设为关闭,勾选\texttt{enable MAC name resolution}。然后开始捕获。

\begin{figure}[H]
  \centering
  \includegraphics[width=0.8\textwidth]{images/02.png}
  \caption{设置\texttt{Wireshark}捕获过滤器}
\end{figure}

回到命令行,再次使用\texttt{ping}命令发起\texttt{ICMP}请求。

\begin{figure}[H]
  \centering
  \includegraphics[width=0.8\textwidth]{images/03.png}
  \caption{再次使用\texttt{ping}命令发起\texttt{ICMP}请求}
\end{figure}

回到\texttt{Wireshark},停止捕获。捕获结果如下图所示:

\begin{figure}[H]
  \centering
  \includegraphics[width=0.8\textwidth]{images/04.png}
  \caption{捕获结果}
\end{figure}

\subsection{分析以太网的帧}

点击捕获到的数据包,选择\texttt{Ethernet II},可以看到\texttt{以太网}的帧结构如下图所示:

\begin{figure}[H]
  \centering
  \includegraphics[width=0.8\textwidth]{images/05.png}
  \caption{\texttt{以太网}的帧结构}
\end{figure}

可以看到\texttt{以太网}头部包括了\texttt{目的地址(Destination)}、\texttt{源地址(Source)}和\texttt{类型(Type)}三部分。其中\texttt{目的地址}和\texttt{源地址}都是\texttt{6}个字节,\texttt{类型}是\texttt{2}个字节。

\begin{figure}[H]
  \centering
  \begin{minipage}[b]{0.42\textwidth}
    \centering
    \includegraphics[width=\textwidth]{images/06.png}
    \caption{\texttt{Destination}}
  \end{minipage}
  \hfill
  \begin{minipage}[b]{0.42\textwidth}
    \centering
    \includegraphics[width=\textwidth]{images/07.png}
    \caption{\texttt{Source}}
  \end{minipage}
\end{figure}

\begin{figure}[H]
  \centering
  \includegraphics[width=0.42\textwidth]{images/08.png}
  \caption{\texttt{Type}}
\end{figure}

画出的帧结构如下图所示:

\begin{figure}[H]
  \centering
  \includegraphics[width=0.8\textwidth]{images/09.png}
  \caption{\texttt{以太网}帧结构}
\end{figure}

\subsection{分析以太网的地址范围}

根据上面分析得到的\texttt{以太网}帧结构,我们可以得知本机\texttt{MAC}地址为\texttt{10:3d:1c:cc:0f:d3},\texttt{IP}地址为\\\texttt{172.30.227.168},路由器\texttt{MAC}地址为\texttt{54:c6:ff:7b:38:02},目标\texttt{IP}地址为\texttt{182.61.200.6}。

可以作出如下的关系图:

\begin{figure}[H]
  \centering
  \includegraphics[width=0.8\textwidth]{images/10.png}
  \caption{\texttt{以太网}地址范围关系图}
\end{figure}

\subsection{分析以太网的广播帧}

启动\texttt{Wireshark},在菜单栏的捕获\(\to \)选项中进行设置,选择已连接的以太网,设置捕获过滤器为\texttt{ether multicast},捕获\texttt{以太网}的广播帧。

\begin{figure}[H]
  \centering
  \includegraphics[width=0.6\textwidth]{images/11.png}
  \caption{设置\texttt{Wireshark}捕获过滤器}
\end{figure}

经过一段时间的捕获,捕获结果如下图所示:

\begin{figure}[H]
  \centering
  \includegraphics[width=0.8\textwidth]{images/12.png}
  \caption{捕获结果}
\end{figure}

选择其中的一个广播帧数据包,如下图所示:

\begin{figure}[H]
  \centering
  \includegraphics[width=0.75\textwidth]{images/13.png}
  \caption{广播帧数据包}
\end{figure}

\begin{enumerate}[noitemsep]
  \item 以太网广播帧的地址是什么,以标准的形式写在Wireshark上显示?

        可以看出,广播帧的地址为\texttt{ff:ff:ff:ff:ff:ff}。

        \begin{figure}[H]
          \centering
          \includegraphics[width=0.75\textwidth]{images/14.png}
          \caption{广播帧地址}
        \end{figure}

  \item 哪几个比特位的以太网地址是用来确定是单播或多播/广播?

        对比单播帧和广播帧,可以看出,\texttt{以太网}地址的第一个字节的最后一位(即第八位)为\texttt{1},所以可以确定是多播/广播。

        \begin{figure}[H]
          \centering
          \begin{minipage}[b]{0.48\textwidth}
            \centering
            \includegraphics[width=\textwidth]{images/16.png}
            \caption{单播帧}
          \end{minipage}
          \hfill
          \begin{minipage}[b]{0.48\textwidth}
            \centering
            \includegraphics[width=\textwidth]{images/15.png}
            \caption{广播帧}
          \end{minipage}
        \end{figure}
\end{enumerate}

\subsection{问题讨论}

设置捕获过滤器为\texttt{llc},捕获\texttt{以太网}的帧,如下图所示:

\begin{figure}[H]
  \centering
  \includegraphics[width=0.75\textwidth]{images/17.png}
  \caption{捕获\texttt{IEEE 802.3 以太网}的帧}
\end{figure}

\begin{enumerate}[noitemsep]
  \item How long are the combined IEEE 802.3 and LLC headers compared to the
        DIX Ethernet headers?

        \textbf{答:}DIX以太网头部长度为\texttt{14}字节,IEEE 802.3头部长度为\texttt{14}字节,LLC头部长度为\texttt{3}字节。如下图所示:

        \begin{figure}[H]
          \centering
          \includegraphics[width=0.75\textwidth]{images/18.png}
          \caption{\texttt{IEEE 802.3 以太网}头部}
        \end{figure}

  \item How does the receiving computer know whether the frame is DIX Ethernet
        or IEEE 802.3?

        \textbf{答:}根据\texttt{Type/Length}字段,如果该字段的值小于或等于\texttt{1500},则表示\texttt{Length},为IEEE 802.3,否则表示\texttt{Type},为DIX以太网。

        \begin{figure}[H]
          \centering
          \includegraphics[width=0.7\textwidth]{images/18.png}
          \caption{\texttt{Length}字段}
        \end{figure}

  \item If IEEE 802.3 has no Type field, then how is the next higher layer
        determined?

        \textbf{答:}\texttt{LLC}头中的\texttt{DSAP}字段可以指示上层协议。例如,此处\texttt{DSAP}字段为\texttt{0x42},则表示上层协议为\texttt{STP}。

        \begin{figure}[H]
          \centering
          \includegraphics[width=0.75\textwidth]{images/19.png}
          \caption{\texttt{DSAP}字段}
        \end{figure}

\end{enumerate}


\section{实验结果总结}

通过本次实验,我学会了通过\texttt{Wireshark}获取\texttt{以太网}的帧,掌握了\texttt{以太网}帧的结构,分析了\texttt{以太网}地址范围,分析了\texttt{以太网}的广播帧。同时,我还了解到了\texttt{DIX 以太网}和\texttt{IEEE 802.3 以太网}的区别。

\section{附录}

无

\end{document}