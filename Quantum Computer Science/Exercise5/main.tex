\documentclass{article}[12pt, a4paper]
\usepackage{amsmath, amssymb}

\title{Exercise 5 (Week 4)}
\author{LI, Pengda \\ 10225101460}
\date{}

\linespread{1.2}

\newcounter{problemname}
\newenvironment{problem}{\stepcounter{problemname}\par\noindent\textsc{\arabic{problemname}.}}{\\\par}
\newenvironment{solution}{\par\noindent\textsc{Solution. }}{\\\\\par}
\newcommand{\qed}{\hfill \square}

\begin{document}
\maketitle

\begin{problem}
    Define the bijective binary-counting function $\mathbb{B}^n \to [0, 2^n)$ and show that its inverse equals the function $bin$ in Section 1.1.
\end{problem}

\begin{solution}
The binary-counting function $\mathbb{B}^n \to [0, 2^n)$ is defined as
$$
f(b_n b_{n-1} \dots b_1) = \sum_{i=1}^{n} 2^{n-i}b_i.
$$

To show that its inverse equals the function $bin$, i.e., $f^{-1}(x) = bin(x)$, we have to prove that
\begin{equation}
    f(bin(x)) = x
\end{equation}
and
\begin{equation}
    bin(f(b_n b_{n-1} \dots b_1)) = b_n b_{n-1} \dots b_1.
\end{equation}

For (1), suppose
$$
bin(x) = b_n b_{n-1} \dots b_1.
$$
By the conversion rule of binary numbers, we have
$$
x = \sum_{i=1}^{n} 2^{n-i}b_i.
$$
Then
$$
f(bin(x)) = f(b_n b_{n-1} \dots b_1) = \sum_{i=1}^{n} 2^{n-i}b_i = x.
$$

For (2), since
$$
f(b_n b_{n-1} \dots b_1) = \sum_{i=1}^{n} 2^{n-i}b_i,
$$
we have
$$
bin(f(b_n b_{n-1} \dots b_1)) = bin(\sum_{i=1}^{n} 2^{n-i}b_i) = b_n b_{n-1} \dots b_1.
$$

By (1) and (2), $f^{-1}(x) = bin(x)$ holds.
$\qed$
\end{solution}

\begin{problem}
    Check the matrix $\Lambda$ in the cases $n = 1,2$.
\end{problem}

\begin{solution}
We have
$$
\begin{aligned}
    \Lambda(\chi) - E(\chi) &= E(\chi) - \chi \\
    \Lambda(\chi) &= 2E(\chi) - \chi \\
    \forall x : \mathbb{B}^n \cdot \Lambda(\chi)(x) &= 2E(\chi) - \chi(x)
\end{aligned}
$$
then for $n = 1$,
$$
\forall x: \mathbb{B} \cdot \Lambda(\chi)(x) = 2(\frac{1}{2}\sum_{y: \mathbb{B}} \chi(y)) - \chi(x),
$$
$$
\Lambda = 2 \times \frac{1}{2}\begin{pmatrix}
1 & 1 \\
1 & 1
\end{pmatrix} - \begin{pmatrix}
    1 & 0 \\
    0 & 1
\end{pmatrix} = \begin{pmatrix}
    0 & 1 \\
    1 & 0
\end{pmatrix},
$$
and for $n = 2$,
$$
\forall x: \mathbb{B}^2 \cdot \Lambda(\chi)(x) = 2(\frac{1}{4}\sum_{y: \mathbb{B}^2} \chi(y)) - \chi(x),
$$
$$
\Lambda =2\times \frac{1}{4}\begin{pmatrix}
1 & 1 & 1 & 1 \\
1 & 1 & 1 & 1 \\
1 & 1 & 1 & 1 \\
1 & 1 & 1 & 1
\end{pmatrix} - \begin{pmatrix}
    1 & 0 & 0 & 0 \\
    0 & 1 & 0 & 0 \\
    0 & 0 & 1 & 0 \\
    0 & 0 & 0 & 1
\end{pmatrix} = \frac{1}{2}\begin{pmatrix}
    -1 & 1 & 1 & 1 \\
    1 & -1 & 1 & 1 \\
    1 & 1 & -1 & 1 \\
    1 & 1 & 1 & -1
\end{pmatrix}.
$$
\end{solution}

\begin{problem}
    Check that multiplication by matrix (1) achieves rotation in
 the plane.
\end{problem}


\begin{solution}
Consider a point represented as $(\cos{b}, \sin{b})$, where $b$ is the angle between the point and the $x$-axis.

Then after rotation by the matrix (1), we have
$$
\begin{aligned}
\begin{pmatrix}
    \cos{a} & -\sin a \\
    \sin a & \cos a
\end{pmatrix}
\begin{pmatrix}
    \cos b \\
    \sin b
\end{pmatrix} &= 
\begin{pmatrix}
    \cos{a}\cos{b} - \sin{a}\sin{b} \\
    \sin{a}\cos{b} + \cos{a}\sin{b}
\end{pmatrix} \\
&= 
\begin{pmatrix}
    \cos{(a+b)} \\
    \sin{(a+b)}
\end{pmatrix},
\end{aligned}
$$
which is the point after rotation by angle $a$.

Therefore, multiplication by matrix (1) achieves rotation in the plane.
\end{solution}

\end{document}