\documentclass{beamer}
\usepackage[UTF8]{ctex}
\usetheme{Boadilla}
% \useoutertheme{shadow}
\useinnertheme{circles}
\usecolortheme{beaver}
% \usefonttheme{structuresmallcapsserif}
\setbeamertemplate{background canvas}[vertical shading][bottom=white,middle=white,top=white]
%\setbeamerfont{title}{size=\LARGE}
%\setbeamercolor{title}{fg=yellow,bg=gray}
%\setbeamertemplate{section in toc}[sections numbered]%节标题模板颜色
%\setbeamercolor{section in toc}{fg=yellow!80!gray}%节标题模板颜色
% \setbeamertemplate{frametitle}{\noindent\insertframetitle\par\noindent\insertframesubtitle\par}
% \setbeamerfont{frametitle}{size=\large}
% \setbeamercolor{frametitle}{fg=yellow!70!gray}
% \setbeamercolor{normal text}{fg=white,bg=gray}
% \setbeamertemplate{blocks}[rounded][shadow=true]
% \setbeamercolor{block title}{fg=white,bg=gray!50!black}
%\setbeamercolor{block body}{bg=gray}
\usepackage[T1]{fontenc}
\usepackage{amsmath}%常用的数学宏包
\usepackage{amssymb}%数学宏包,可以实现一些数学符号
\usepackage{graphicx}
\usepackage{float}
\RequirePackage{caption}
\usepackage{subfigure}
\usepackage{fancyhdr}
\usepackage{listings}
\usepackage{pxfonts}
\usepackage{geometry}
\usepackage{setspace}
\usepackage{amsfonts}%数学字体宏包
\usepackage{comment}%批量注释的宏包
\usepackage{boxedminipage}%为文本段添加边框的宏包
\usepackage{shadow}%带阴影的边框
\usepackage{fancybox}%实现边框效果的宏包
\usepackage{esint}%实现多重积分的宏包
\usepackage{relsize}%提供加大数学符号命令\mathlarger
\usepackage{listings}

\setmonofont{JetBrains Mono}
\setCJKmonofont{黑体}

\definecolor{lightgray}{rgb}{.9,.9,.9}
\definecolor{darkgray}{rgb}{.4,.4,.4}
\definecolor{purple}{rgb}{0.65, 0.12, 0.82}

\lstdefinelanguage{JavaScript}{
  keywords={typeof, new, true, false, catch, function, return, null, catch, switch, var, if, in, while, do, else, case, break, type, interface, implements, class, export, boolean, throw, implements, import, this},
  keywordstyle=\color{blue}\bfseries,
  ndkeywords={class, export, boolean, throw, implements, import, this},
  ndkeywordstyle=\color{darkgray}\bfseries,
  identifierstyle=\color{black},
  sensitive=false,
  comment=[l]{//},
  morecomment=[s]{/*}{*/},
  commentstyle=\color{purple}\ttfamily,
  stringstyle=\color{red}\ttfamily,
  morestring=[b]',
  morestring=[b]"
}

\lstset{
  % language = C,
  xleftmargin = 1em,xrightmargin = 1em, aboveskip = 1em,
	backgroundcolor = \color{white}, % 背景色
	basicstyle = \footnotesize\ttfamily, % 基本样式 + 小号字体
	rulesepcolor= \color{gray}, % 代码块边框颜色
	breaklines = true, % 代码过长则换行
	numbers = left, % 行号在左侧显示
	numberstyle = \small, % 行号字体
  numbersep = -14pt, 
  keywordstyle=\color{purple}\bfseries, % 关键字颜色
  commentstyle =\color{red!50!green!50!blue!60}, % 注释颜色
  stringstyle = \color{red}, % 字符串颜色
  morekeywords={ASSERT, int64_t, uint32_t},
	frame = shadowbox, % 用(带影子效果)方框框住代码块
	showspaces = false, % 不显示空格
	columns = fixed, % 字间距固定
} 

\AtBeginSection[]{
  \begin{frame}
  \vfill
  \centering
  \begin{beamercolorbox}[sep=8pt,center,shadow=true,rounded=true]{title}
    \usebeamerfont{title}\insertsectionhead\par%
  \end{beamercolorbox}
  \vfill
  \end{frame}
}

\title{\texttt{React}前端开发}
\subtitle{第四讲 \texttt{Effect}、\texttt{Context}、其他补充内容与常用库}
\author{PDLi}
\date{2023年12月17日}

\begin{document}
\begin{frame}
  \titlepage
\end{frame}
\begin{frame}
  \frametitle{目录}
  \tableofcontents
\end{frame}

\section{\texttt{Effect}}
\begin{frame}[allowframebreaks, fragile]
  \frametitle{\texttt{Effect}}

  在理想状态下,我们希望 \texttt{React} 组件的渲染结果只与 \texttt{props} 和 \texttt{state} 相关,但是实际上,我们经常需要在组件渲染的时候执行一些额外的操作,比如获取数据、订阅事件等。这些操作被称为 \texttt{side effect(副作用)},\texttt{React} 为我们提供了 \texttt{useEffect Hook} 来处理这些 \texttt{side effect}。

  \framebreak

  \begin{block}{\texttt{useEffect}}
    \texttt{useEffect} 接收一个函数作为参数,这个函数就是我们需要执行的副作用。在组件渲染的时候,\texttt{React} 会保存这个函数,然后在组件渲染完成后执行这个函数。如果我们需要在组件卸载的时候执行一些清理操作,我们可以在这个函数中返回一个函数,\texttt{React} 会在组件卸载的时候执行这个函数。
    \texttt{useEffect}接受一个可选的第二个参数,这个参数是一个数组,数组中的每个元素都是一个依赖项。如果依赖项发生了变化,\texttt{React} 会重新执行这个副作用函数。如果没有传入依赖项,那么每次组件渲染的时候都会执行这个副作用函数。
  \end{block}

  \framebreak

  示例代码:
  \begin{lstlisting}[language=JavaScript]
    import React, { useState, useEffect } from 'react';

    function Example() {
      const [count, setCount] = useState(0);
      const [data, setData] = useState('');

      useEffect(() => {
        // 使用 setTimeout 模拟异步请求
        setTimeout(() => {
          setData('111111');
        }, 1000);
      }, []);

      return (
        <div>
          <p>You clicked {count} times</p>
          <h1>{data}</h1>
          <button onClick={() => setCount(count + 1)}>
            Click me
          </button>
        </div>
      );
    }
  \end{lstlisting}

  设置依赖项:

  \begin{lstlisting}[language=JavaScript]
    import React, { useState, useEffect } from 'react';

    function Example() {
      const [count, setCount] = useState(0);

      useEffect(() => {
        document.title = `You clicked ${count} times`;
      }, [count]);

      return (
        <div>
          <p>You clicked {count} times</p>
          <button onClick={() => setCount(count + 1)}>
            Click me
          </button>
        </div>
      );
    }
  \end{lstlisting}

  设置定时器:

  \begin{lstlisting}[language=JavaScript]
    import React, { useState, useEffect } from 'react';

    function Example() {
      const [count, setCount] = useState(0);

      useEffect(() => {
        const timer = setInterval(() => {
          setCount(count + 1);
        }, 1000);
        return () => clearInterval(timer);
      }, [count]);

      return (
        <div>
          <p>You clicked {count} times</p>
        </div>
      );
    }
  \end{lstlisting}

  \framebreak

  \begin{block}{\texttt{useEffect}的执行时机}
    \begin{itemize}
      \item \texttt{React} 会在每次渲染后调用副作用函数,包括第一次渲染的时候。
      \item 之后每次渲染前,\texttt{React} 会先清除上一次渲染时的副作用函数。
      \item \texttt{React} 会在组件卸载的时候执行副作用函数,清除副作用。
    \end{itemize}
  \end{block}

  \begin{alertblock}{依赖项的设置}
    \begin{itemize}
      \item 如果没有传入依赖项,那么每次组件渲染的时候都会执行这个副作用函数。
      \item 如果传入了空数组,那么副作用函数只会在组件渲染的时候执行一次,之后不会再执行。
      \item 如果传入了非空数组,那么副作用函数会在组件渲染的时候执行一次,之后只有当依赖项发生变化的时候才会执行。
    \end{itemize}
  \end{alertblock}


\end{frame}

\section{\texttt{Context}}

\begin{frame}[allowframebreaks, fragile]
  \frametitle{\texttt{Context}}

  \texttt{Context} 提供了一种在组件之间共享值的方式,而不必通过组件树的逐层传递 \texttt{props}。在一些场景下,这种做法会使得组件的传递变得很复杂,\texttt{Context} 可以帮助我们解决这个问题。

  \framebreak

  \begin{block}{\texttt{Context}的使用}
    \begin{itemize}
      \item 首先,我们需要创建一个 \texttt{Context} 对象,这个对象包含一个 \texttt{Provider} 组件和一个 \texttt{Consumer} 组件。
      \item 然后,我们需要在 \texttt{Provider} 组件中传入一个值,这个值可以是任意类型的数据。
      \item 最后,我们可以在 \texttt{Consumer} 组件中获取到 \texttt{Provider} 组件中传入的值。
    \end{itemize}
  \end{block}

  \framebreak

  示例代码:
  \begin{lstlisting}[language=JavaScript]
    import React, { useState, useEffect, createContext, useContext } from 'react';

    const CountContext = createContext();

    function Counter() {
      const count = useContext(CountContext);
      return <h2>{count}</h2>;
    }

    function Example() {
      const [count, setCount] = useState(0);

      useEffect(() => {
        const timer = setInterval(() => {
          setCount(count + 1);
        }, 1000);
        return () => clearInterval(timer);
      }, [count]);

      return (
        <div>
          <p>You clicked {count} times</p>
          <CountContext.Provider value={count}>
            <Counter />
          </CountContext.Provider>
        </div>
      );
    }
  \end{lstlisting}

\end{frame}

\section{其他补充内容}
\subsection{\texttt{JSX} 中的 \texttt{if}语句}
\begin{frame}[allowframebreaks, fragile]
  \frametitle{在 \texttt{JSX} 中使用 \texttt{if} 语句}

  \begin{block}{\texttt{if} 语句}
    \texttt{JSX} 本身并不支持 \texttt{if} 语句,但是我们可以在 \texttt{JSX} 中使用三元运算符或 \texttt{\&\&} 来实现 \texttt{if} 语句的功能。
  \end{block}

  \framebreak

  示例代码:

  \begin{lstlisting}[language=JavaScript]
    import React, { useState } from 'react';

    function Example() {
      const [count, setCount] = useState(0);

      return (
        <div>
          <p>You clicked {count} times</p>
          {count % 2 === 0 ? <h1>偶数</h1> : <h1>奇数</h1>}
          {count > 5 && <h1>大于5</h1>}
          <button onClick={() => setCount(count + 1)}>
            Click me
          </button>
        </div>
      );
    }
  \end{lstlisting}

\end{frame}

\subsection{\texttt{JSX} 中的 \texttt{for} 循环}
\begin{frame}[allowframebreaks, fragile]
  \frametitle{在 \texttt{JSX} 中使用 \texttt{for} 循环}

  \begin{block}{\texttt{for} 循环}
    \texttt{JSX} 本身并不支持 \texttt{for} 循环,但是我们可以在 \texttt{JSX} 中使用 \texttt{map} 来实现 \texttt{for} 循环的功能。
  \end{block}

  \framebreak

  示例代码:

  \begin{lstlisting}[language=JavaScript]
    import React, { useState } from 'react';

    function Example() {
      const [count, setCount] = useState(0);

      return (
        <div>
          <p>You clicked {count} times</p>
          {[1, 2, 3, 4, 5].map((item) => {
            return <h1 key={item}>{item}</h1>;
          })}
          <button onClick={() => setCount(count + 1)}>
            Click me
          </button>
        </div>
      );
    }
  \end{lstlisting}

  \framebreak

  \begin{alertblock}{\texttt{key} 属性}
    在使用 \texttt{map} 的时候,我们需要为每个元素添加一个 \texttt{key} 属性,这个属性的值应该是唯一的,这样 \texttt{React} 才能够正确地识别每个元素。

    \texttt{key} 属性的作用是帮助 \texttt{React} 识别哪些元素发生了变化,从而减少 \texttt{DOM} 操作的次数,提高性能。
  \end{alertblock}

\end{frame}


\subsection{受控组件与非受控组件}
\begin{frame}[allowframebreaks, fragile]
  \frametitle{受控组件与非受控组件}

  \begin{block}{受控组件}
    受控组件是指表单元素的值由 \texttt{React} 组件来控制的组件,比如下面的代码:
    \begin{lstlisting}[language=JavaScript]
    import React, { useState } from 'react';

    function Example() {
      const [value, setValue] = useState('');

      return (
        <div>
          <input value={value} onChange={(e) => setValue(e.target.value)} />
        </div>
      );
    }
  \end{lstlisting}
  \end{block}

  \framebreak

  \begin{block}{非受控组件}
    非受控组件是指表单元素的值由 \texttt{DOM} 元素来控制的组件,比如下面的代码:
    \begin{lstlisting}[language=JavaScript]
    import React, { useRef } from 'react';

    function Example() {
      const inputRef = useRef(null);

      return (
        <div>
          <input ref={inputRef} />
        </div>
      );
    }
  \end{lstlisting}
  \end{block}

  \framebreak

  \begin{alertblock}{受控组件与非受控组件}
    \begin{itemize}
      \item 受控组件的值由 \texttt{React} 组件来控制,所以我们可以通过 \texttt{props} 来控制表单元素的值。
      \item 非受控组件的值由 \texttt{DOM} 元素来控制,所以我们只能通过 \texttt{DOM} 元素的方法来控制表单元素的值。
    \end{itemize}
  \end{alertblock}

  \begin{block}{使用 \bf{受控组件}}
    在一般情况下,我们应该尽量使用受控组件,因为这样可以让表单元素的值与 \texttt{React} 组件的状态保持一致,从而减少出错的可能性。
  \end{block}

\end{frame}

\section{常用库}
\subsection{\texttt{React Router}}
\subsubsection{\texttt{React Router} 简介与安装}
\begin{frame}[allowframebreaks, fragile]
  \frametitle{\texttt{React Router} 简介与安装}

  \begin{block}{\texttt{React Router}}
    \texttt{React Router} 是一个用于 \texttt{React} 的路由库,它可以帮助我们实现页面之间的跳转。
  \end{block}

  通过 \texttt{React Router},我们可以在 \texttt{SPA} 应用中实现页面之间的跳转,而不需要刷新页面。

  \framebreak

  \begin{block}{安装}
    \begin{lstlisting}[language=bash]
    npm install react-router-dom
    \end{lstlisting}
  \end{block}
\end{frame}

\subsubsection{\texttt{ReactRouter} 的使用}
\begin{frame}[allowframebreaks, fragile]
  \frametitle{\texttt{ReactRouter} 的使用}

  \begin{block}{\texttt{ReactRouter} 的使用}
    \begin{itemize}
      \item 首先,我们需要在 \texttt{App.js} 中导入 \texttt{BrowserRouter} 组件,然后将 \texttt{App} 组件包裹在 \texttt{BrowserRouter} 组件中。
      \item 然后,我们需要在 \texttt{App.js} 中导入 \texttt{Routes} 组件,然后在 \texttt{BrowserRouter} 组件中添加 \texttt{Routes} 组件。
      \item 最后,我们需要在 \texttt{Routes} 组件中添加 \texttt{Route} 组件,其中 \texttt{path} 属性表示路由的路径,\texttt{element} 属性表示路由对应的组件。
    \end{itemize}
  \end{block}

  \framebreak

  示例代码:

  \begin{lstlisting}[language=JavaScript]
    import React from 'react';
    import { BrowserRouter, Routes, Route } from 'react-router-dom';
    import Home from './Home';
    import About from './About';

    function App() {
      return (
        <BrowserRouter>
          <Routes>
            <Route path="/" element={<Home />} />
            <Route path="/about" element={<About />} />
          </Routes>
        </BrowserRouter>
      );
    }

    export default App;
  \end{lstlisting}

  \framebreak

  \begin{alertblock}{\texttt{Route} 组件的属性}
    \begin{itemize}
      \item \texttt{path} 属性表示路由的路径。
      \item \texttt{element} 属性表示路由对应的组件。
      \item \texttt{caseSensitive} 属性表示是否区分大小写。
      \item \texttt{index} 属性表示索引路由。
      \item \texttt{children} 属性表示嵌套路由。
    \end{itemize}
  \end{alertblock}

  \framebreak

  \begin{block}{\texttt{caseSensitive} 属性}
    \texttt{caseSensitive} 属性表示是否区分大小写,如果设置了 \texttt{caseSensitive} 属性,那么只有当路径大小写完全匹配的时候才会渲染对应的组件。
  \end{block}

  示例代码:

  \begin{lstlisting}[language=JavaScript]
    import React from 'react';
    import { BrowserRouter, Routes, Route } from 'react-router-dom';
    import Home from './Home';
    import About from './About';

    function App() {
      return (
        <BrowserRouter>
          <Routes>
            <Route path="/" element={<Home />} />
            <Route path="/about" element={<About />} />
            <Route path="/About" element={<About />} caseSensitive />
          </Routes>
        </BrowserRouter>
      );
    }

    export default App;
  \end{lstlisting}

  \framebreak

  \begin{block}{\texttt{index} 属性}
    \texttt{index} 属性表示索引路由,如果设置了 \texttt{index} 属性,那么这个组件将被渲染到父级路由的 \texttt{Outlet} 中。
  \end{block}

  \begin{block}{嵌套路由}
    我们可以在 \texttt{Route} 组件下添加子组件来实现嵌套路由。
  \end{block}

  示例代码:

  \begin{lstlisting}[language=JavaScript]
    import React from 'react';
    import { BrowserRouter, Routes, Route } from 'react-router-dom';
    import Home from './Home';
    import About from './About';
    import AboutMe from './AboutMe';
    import AboutYou from './AboutYou';

    function App() {
      return (
        <BrowserRouter>
          <Routes>
            <Route path="/" element={<Home />} />
            <Route path="/about" element={<About />}>
              <Route path="me" element={<AboutMe />} />
              <Route path="you" element={<AboutYou />} />
            </Route>
          </Routes>
        </BrowserRouter>
      );
    }

    export default App;
  \end{lstlisting}

\end{frame}

\subsubsection{路由表}
\begin{frame}[allowframebreaks, fragile]
  \frametitle{路由表}

  \begin{block}{路由表}
    我们可以将路由表单独抽离出来,然后在 \texttt{App.js} 中导入路由表。并使用 \texttt{useRoutes} 函数来渲染路由表。
  \end{block}

  示例代码:

  \begin{lstlisting}[language=JavaScript]
    import React from 'react';
    import { BrowserRouter as Router, Route, useRoutes } from 'react-router-dom';
    import Home from './Home';
    import About from './About';
    import AboutMe from './AboutMe';
    import AboutYou from './AboutYou';

    const routes = [
      {
        path: '/',
        element: <Home />,
      },
      {
        path: '/about',
        element: <About />,
      },
      {
        path: '/about/me',
        element: <AboutMe />,
      },
      {
        path: '/about/you',
        element: <AboutYou />,
      },
    ];

    function App() {
      const routing = useRoutes(routes);

      return (
        <Router>
          {routing}
        </Router>
      );
    }

    export default App;
  \end{lstlisting}

\end{frame}

\subsubsection{路由参数}

\begin{frame}[allowframebreaks, fragile]
  \frametitle{路由参数}

  \begin{block}{路由参数}
    我们可以在路由中添加参数,然后在组件中通过 \texttt{useParams} 或 \texttt{useSearchParams} 函数来获取路由参数。
  \end{block}

  示例代码:

  \begin{lstlisting}[language=JavaScript]
    import React from 'react';
    import { BrowserRouter as Router, Route, useRoutes, useParams } from 'react-router-dom';
    import Home from './Home';
    import About from './About';
    import AboutMe from './AboutMe';
    import AboutYou from './AboutYou';

    const routes = [
      {
        path: '/',
        element: <Home />,
      },
      {
        path: '/about',
        element: <About />,
      },
      {
        path: '/about/me',
        element: <AboutMe />,
      },
      {
        path: '/about/you',
        element: <AboutYou />,
      },
      {
        path: '/about/:name',
        element: <AboutName />,
      },
    ];

    function AboutName() {
      const { name } = useParams();
      return <h1>{name}</h1>;
    }

    function App() {
      const routing = useRoutes(routes);

      return (
        <Router>
          {routing}
        </Router>
      );
    }

    export default App;
  \end{lstlisting}

  \texttt{search} 参数:

  \begin{lstlisting}[language=JavaScript]
    import React from 'react';
    import { BrowserRouter as Router, Route, useRoutes, useSearchParams } from 'react-router-dom';
    import Home from './Home';
    import SearchResults from './SearchResults';
    
    const routes = [
      {
        path: '/',
        element: <Home />,
      },
      {
        path: '/search',
        element: <SearchResults />,
      },
    ];
    
    function SearchResults() {
      const [searchParams] = useSearchParams();
      const query = searchParams.get('query');
    
      return <h1>Search Results for: {query}</h1>;
    }
    
    function App() {
      const routing = useRoutes(routes);
    
      return (
        <Router>
          {routing}
        </Router>
      );
    }
    
    export default App;
\end{lstlisting}

\end{frame}

\subsubsection{\texttt{Link} 组件}

\begin{frame}[allowframebreaks, fragile]
  \frametitle{\texttt{Link} 组件}

  \begin{block}{\texttt{Link} 组件}
    \texttt{Link} 组件可以帮助我们实现页面之间的跳转。
  \end{block}

  示例代码:

  \begin{lstlisting}[language=JavaScript]
    import React from 'react';
    import { BrowserRouter as Router, Route, useRoutes, Link } from 'react-router-dom';
    import Home from './Home';
    import About from './About';
    import AboutMe from './AboutMe';
    import AboutYou from './AboutYou';

    const routes = [
      {
        path: '/',
        element: <Home />,
      },
      {
        path: '/about',
        element: <About />,
      },
      {
        path: '/about/me',
        element: <AboutMe />,
      },
      {
        path: '/about/you',
        element: <AboutYou />,
      },
      {
        path: '/about/:name',
        element: <AboutName />,
      },
    ];

    function AboutName() {
      const { name } = useParams();
      return <h1>{name}</h1>;
    }

    function App() {
      const routing = useRoutes(routes);

      return (
        <Router>
          <nav>
            <Link to="/">Home</Link>
            <Link to="/about">About</Link>
            <Link to="/about/me">About Me</Link>
            <Link to="/about/you">About You</Link>
          </nav>
          {routing}
        </Router>
      );
    }

    export default App;
  \end{lstlisting}

\end{frame}

\end{document}